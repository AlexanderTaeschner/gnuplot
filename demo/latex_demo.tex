% Illustrate use of alternative LaTeX-based gnuplot terminals
% Shows
%   pict2e
%   cairolatex pdf
%   tikz
% Mentions
%   texdraw pstex pstricks
%   epslatex pslatex
%   context
% but not all are included in this demo.

% Page layout
\documentclass[letterpaper,11pt]{article}
\usepackage[margin=0.75in]{geometry}
\usepackage[parfill]{parskip}
\usepackage[T1]{fontenc}
\usepackage{textcomp}
\usepackage{framed}

% straight single quotes
\usepackage{upquote}

% symbols
\usepackage{latexsym}
\usepackage{amssymb}

% colors
\usepackage{graphicx}
\usepackage{xcolor}

% pict2e terminal
\usepackage{pict2e}
\usepackage{transparent}

% tikz terminal
\usepackage{gnuplot-lua-tikz}

% needed for pdflatex but not for xelatex or lualatex
% \usepackage[utf8x]{inputenc}
% \SetUnicodeOption{mathletters}

% figure insertion mechanism via \usepackage{wrapfig}
% I personally prefer the picins package, but this is no longer
% guaranteed to be present in current TeX distributions.
% \usepackage{wrapfig}
% \newcommand{\gpinsetfigure}[2]{
%    \begin{wrapfigure}[10]{r}{3.5in}
%    \vspace{-20pt} \input{#1} \vspace{-20pt}
%    \caption{#2}
%    \end{wrapfigure}
% }

% alternative figure insertion mechanism using a plain float
\usepackage{float}
\newcommand{\gpinsetfigure}[2]{
  \begin{figure}[H]
  \centering
  \input{#1}
  \caption{#2}
  \end{figure}
}

\begin{document}

\title {\vspace{-.75in} \TeX-friendly gnuplot terminals \vspace{-5pt}}
\author{Ethan A Merritt - January 2022}
\date{}
\maketitle

This demo shows various options for combining gnuplot graphics with 
{\LaTeX} processing of text, usually to produce a figure that is to be
included in a larger {\LaTeX} document.
You have a choice among several \TeX-friendly gnuplot terminals.
Some of these ({\em pict2e}, {\em tikz}) hand off both the graphics
and the text to be processed by latex or pdflatex.
The {\em cairolatex} terminal uses the cairo graphics library to produce
a file containing the graphics in *.png, *.eps or *.pdf format and a
second parallel file *.tex containing the accompanying {\TeX} text elements.
The {\em pslatex} and {\em epslatex} terminals similarly create a pair of
files but the graphics file is limited to PostScript (*.ps or *.eps).

Although all of these choices use a {\TeX} or {\LaTeX} variant to process the
text component of the figure, the details vary.  For example,
some terminals optionally also use {\TeX} to draw individual point
symbols rather than treating them as graphics elements. Some
offer a similar option for treating arrows as {\TeX} elements rather
than as graphics.

Some terminals, notably {\em cairolatex} and {\em tikz}, offer a
"standalone" option to produce a self-contained {\LaTeX} document
rather than a fragment for inclusion in a larger document.

All of the figures shown here were produced with the same gnuplot code,
changing only the terminal type.  Here are the gnuplot commands

\begin{framed}
\begin{minipage}{\textwidth}
\begin{verbatim}
  # set title "It would be nice if this were converted into a caption!"
  set xtics 0.5 nomirror
  set tics format "%.1f"
  set margins -1,0,7,0
  set xrange [-1:1]
  set yrange [1:3]
  set key notitle invert under reverse Left left spacing 2 samplen 0.7
  set arrow 1 filled from graph 0.4, 0.7 to graph 0.6, 0.7
  set label 1 at graph 0.5, 0.75 "$k$" center
  Title_E = 'EllipticE$(k)=\int_0^{\pi/2} {\sqrt{1-k^2\sin^2\theta}}~d\theta$'
  Title_K = 'EllipticK$(k)=\int_0^{\pi/2} {\sqrt{1-k^2\sin^2\theta}~}^{-1}~d\theta$'
  
  plot EllipticE(x) lw 3 title Title_E,  EllipticK(x) lw 3 title Title_K
\end{verbatim}
\end{minipage}
\end{framed}

\noindent
Things to note
\begin{itemize}
\item{All text including plot titles is provided in {\TeX} syntax, not in
      gnuplot's own markup syntax. However formats as in the {\tt set tics}
      command are applied before output, so {\TeX} never sees the
      {\tt \%} character.}
\item{Even though we request the same font size and the same overall size
      for all terminals, it doesn't come out exactly that way.}
\item{All of the latex terminals, except maybe tikz, do a poor job of choosing
      the margins automatically.  It is a good idea to set them explicitly.}
\item{There is no way I know of to automatically transform the text from
      gnuplot {\tt set title "text"}
      into a corresponding latex {\tt \verb+\caption{text}+}.}
      The captions you see here were added explicitly in the outer
      latex\_demo.tex document.
\end{itemize}

\newpage

\subsection*{pict2e terminal}

This version of the figure is produced by the {\em pict2e} gnuplot terminal.
This terminal was created by Bastian Märkisch based on the original gnuplot
{\em latex} terminal written by David Kotz and its successor terminals
{\em emtex}, {\em eepic}, and {\em tpic}.
The {\em pict2e} terminal requires standard latex support packages
{\em pict2e xcolor graphicx amssymb}.  To support transparent fill areas
when processed through pdflatex it also requires package {\em transparent}.
Both the graphics and the text are processed by latex, or in the case of this
demo by pdflatex or xelatex.

\begin{verbatim}
set terminal pict2e color texarrows font "cmr,10" size 3.5in,2.4in
\end{verbatim}
\gpinsetfigure{latex_pict2e}
              {Complete elliptic integrals of the first and second kinds}



\subsection*{cairolatex terminal}

The {\em cairolatex} terminal uses the cairo graphics library to produce
the graphical parts of the figure as a *.eps *.pdf or *.png file (your choice),
and it produces a parallel *.tex file so that latex (or pdflatex, xelatex, ...)
can combine the graphics and text into a single figure for later inclusion in a
latex document.  Generally you would use the {\tt eps} option only when working
in a plain latex/dvips environment that is limited to PostScript file inclusion,
and use the {\tt png} option only when the figure includes pixel-based images.

\begin{verbatim}
set terminal cairolatex color pdf font "cmr,10" fontscale 0.7 size 3.5in,2.4in
\end{verbatim}
\gpinsetfigure{latex_cairo}
              {Complete elliptic integrals of the first and second kinds}


\subsection*{tikz terminal}

The {\em tikz} terminal is the option I usually recommend.
It is simple to use although it is complex underneath, as the gnuplot
terminal driver interfaces with an external lua script to produce a *.tex file
containing both text and graphics commands for the PGF and TikZ {\TeX} packages.

\begin{verbatim}
set terminal tikz size 3.5in,2.4in
\end{verbatim}
\gpinsetfigure{latex_tikz}
              {Complete elliptic integrals of the first and second kinds}

By default it uses {\TeX} arrows, but uses gnuplot point symbols.
To see the many terminal options available, tell gnuplot {\tt set term tikz help}.
The options {\tt tex}, {\tt latex} and {\tt context} tailor the output for
a specific {\TeX} environment.  {\tt latex} is the default.
Load the corresponding style file at the beginning of your document
\begin{verbatim}
   \input gnuplot-lua-tikz.tex    % (for plain TeX)
   \usepackage{gnuplot-lua-tikz}  % (for LaTeX)
   \usemodule[gnuplot-lua-tikz]   % (for ConTeXt)
\end{verbatim}
Other common requirements
\begin{verbatim}
   \usepackage{latexsym}
   \usepackage{amssymb}
   \usepackage[utf8x]{inputenc}   % needed for pdflatex but not xelatex or lualatex
   \SetUnicodeOption{mathletters} % needed for pdflatex but not xelatex or lualatex
\end{verbatim}

The tikz terminal can also be used to create a pdf file, either for use by
itself of for inclusion as a figure in a larger document.  Note that in this
case text is rendered at the time you process the gnuplot output to produce
a pdf file; i.e. if you intend to include this pdf file in a {\TeX} document
you must take care to choose the size and font to match the local environment
of the surrounding document.
The example below uses pdflatex, but lualatex or xelatex should also work. 
\begin{verbatim}
   gnuplot> set term tikz standalone size 10cm,7cm font='\small'
   gnuplot> set output 'figure.tex'
   gnuplot> ... various plot commands ...
   gnuplot> unset output
   gnuplot> !pdflatex figure
\end{verbatim}

\newpage

\subsection*{context terminal}

The {\em context} terminal produces Metafun code for use by the 
ConTeXt macro package. Typically this terminal is selected as part of a stream
of gnuplot commands generated inside a ConTeXt session; the resulting output
is then fed back into that session.
For more information, see {\em ctan.org/pkg/context-gnuplot}.

\subsection*{Other terminals: pstex epslatex pslatex}

The {\em pstex}, {\em epslatex}, and {\em pslatex} terminals are perhaps
best viewed as serving a slightly different purpose.
Rather than aiding inclusion of gnuplot graphs in a {\LaTeX} document,
they are primarily intended to enable {\TeX}-set annotation of gnuplot graphs
included in a PostScript document.  The toolchain in this case is typically
{$ gnuplot \rightarrow latex \rightarrow dvips \rightarrow document.ps$}.

Nevertheless it is possible to use these terminals to create a standalone
*.eps file containing both graphics and already-rendered {\TeX} content,
and then include that *.eps file in a {\LaTeX} document using a
{\em {\verb+\includegraphics{}+}} statement as shown here.

\begin{verbatim}
set terminal epslatex color 10 size 3.5in,2.4in standalone
\end{verbatim}

\begin{figure}[H]
   \centering
   \includegraphics{latex_epslatex.eps}
\end{figure}


\subsection*{Legacy terminals}

Two older gnuplot terminals {\em pstricks} and {\em texdraw} remain in the
default build configuration.  They offer limited graphics capability
compared to the terminals described above.

Other older gnuplot terminals {\em latex}, {\em eepic}, {\em emtex}, and {\em tpic}
have been deprecated and are no longer included by default when gnuplot is built.
If necessary, the file {\em src/term.h} can be edited to uncomment {\em \#include}
statements for the corresponding terminal driver files prior to building gnuplot.

The {\em metapost} and {\em metafont} terminals have also been deprecated.
They can be including when building gnuplot by adding configuration options
{\tt --with-metapost --with-metafont}.


\end{document}
