\documentclass[a4paper,10pt]{article}
\usepackage[T1]{fontenc}
\usepackage{booktabs}
\usepackage{longtable}
\usepackage{graphicx}
\usepackage{array}
%\usepackage[width=18cm,height=24cm]{geometry}
\addtolength{\textwidth}{30mm}
\addtolength{\oddsidemargin}{-15mm}

\font\cmmi=cmmi10
\font\cmr=cmr10
\font\cmti=cmti10
\font\cmtt=cmtt10
\font\cmu=cmu10
\font\cmsy=cmsy10
\font\cmex=cmex10
\font\cmss=cmss10
\font\cmff=cmff10
\font\cmtex=cmtex10
\font\lasy=lasy10

\newcount\rownum

\newcommand*\row[1]{%
  \textbackslash #1 & 
  {\cmr\char'#1} & 
  {\cmti\char'#1} & 
  {\cmtt\char'#1} & 
  {\cmmi\char'#1} &
  {\cmu\char'#1} &
  {\cmss\char'#1} &
  {\cmtex\char'#1} &
  {\cmff\char'#1} &
  {\cmsy\char'#1} &
  {\lasy\char'#1} &
  \ifodd\rownum \else\qquad\fi
  \global\advance\rownum 1
  {\raisebox{1.7ex}[0mm][1ex]{\cmex\char'#1}} &
  \textbackslash #1 \\
}

\begin{document}
\title{Using \TeX\ Fonts in the Gnuplot Postscript Terminal}
\author{Harald Harders, {\ttfamily h.harders@tu-bs.de}}
\date{2002-10-15}
\maketitle

The Postscript terminal can embed Postscript Type\,1 fonts (with
extensions \verb|.pfa| and \verb|.pfb|) and TrueType fonts (extension
\verb|.ttf|) using the command
\begin{verbatim}
set terminal postscript fontfile '<filename>'
\end{verbatim}
The \verb|fontfile| option can be used multiple times.
See the sections \emph{set terminal postscript} and \emph{set
  fontpath} in the Gnuplot documentation for further description.

The embedded font can be used by 
\begin{verbatim}
set terminal postscript '<fontname>' <size>
\end{verbatim}
or in postscript enhanced terminal as following example:
\begin{verbatim}
set xlabel '{/CMMI10 x}'
\end{verbatim}

The font embedding is useful for generating plots to be included in
\LaTeX\ documents. 
For normal text, the \emph{cm-super} Postscript Type\,1 fonts are a
good choice. 
They are available from CTAN servers, e.g.
\begin{verbatim}
ftp://ftp.dante.de/tex-archive/fonts/ps-type1/cm-super/
\end{verbatim}
The normal upright font with serifes is defined in
\verb|sfrm1000.pfb|, and the font name is \verb|SFRM1000|\footnote{If you
have an old version of the cm-super font, prior 2001-10-14, the font name is
in lowercase letters: \texttt{sfrm1000}.} (The
\verb|1000| means that this font is designed for 10\,pt).
Replace the \verb|rm| by \verb|it|, \verb|bx| or other combinations in
both the file name and the font name (here, in uppercase letters) in order 
to get other font shapes.
The encoding of these fonts is ordinary and thus is not described
here.
Table~\ref{tab:fontnames} shows some examples of fonts contained in
the cm-super font bundle.
%
\begin{table}
  \centering
  \begin{tabular}{>{\ttfamily}l>{\ttfamily}ll}
    \toprule
    \multicolumn{1}{l}{File name}&
    \multicolumn{1}{l}{Full font name} &
    Example \\
    & \multicolumn{1}{l}{(all preceded by \texttt{Computer Modern})} & \\
    \midrule
    sfrm1000.pfb& Roman &
    {\rmfamily\upshape Example} \\
    sfbx1000.pfb& Bold Extended &
    {\rmfamily\bfseries\upshape Example} \\
    sfti1000.pfb& Italic &
    {\rmfamily\itshape Example} \\
    sfbi1000.pfb& Bold Extended Italic &
    {\rmfamily\bfseries\itshape Example} \\
    sfsl1000.pfb& Slanted &
    {\rmfamily\slshape Example} \\
    sfbl1000.pfb& Bold Extended Slanted &
    {\rmfamily\bfseries\slshape Example} \\
    sfcc1000.pfb& Caps and Small Caps &
    {\rmfamily\bfseries\scshape Example} \\
    \midrule
    sfss1000.pfb& Sans Serif &
    {\sffamily\upshape Example} \\
    sfsi1000.pfb& Sans Serif Slanted &
    {\sffamily\slshape Example} \\
    sfsx1000.pfb& Sans Serif Bold Extended &
    {\sffamily\bfseries\upshape Example} \\
    sfso1000.pfb& Sans Serif Bold Extended Slanted &
    {\sffamily\bfseries\slshape Example} \\
    \midrule
    sftt1000.pfb& Typewriter &
    {\ttfamily\upshape Example} \\
    sfit1000.pfb& Typewriter Italic &
    {\ttfamily\itshape Example} \\
    sfst1000.pfb& Typewriter Slanted &
    {\ttfamily\slshape Example} \\
    sftc1000.pfb& Typewriter Caps and Small Caps &
    {\ttfamily\scshape Example} \\
    \bottomrule
  \end{tabular}
  \caption{Some fonts in the cm-super font bundle (for a designsize of
    10\,pt)}%
  \label{tab:fontnames}%
\end{table}

For mathematics the Type\,1 versions of the Computer Modern fonts are
useful.
They should be installed in most \TeX\ implementations and are also
available from CTAN servers, e.g.
\begin{verbatim}
ftp://ftp.dante.de/tex-archive/fonts/cm/ps-type1/bluesky/pfb/
\end{verbatim}
Here, the font name is the base of the file name in uppercase letters,
e.g. the file \verb|cmmi10.pfb| contains the font \verb|CMMI10|.
Since the encoding of these fonts is strange, a table containing all
characters for some fonts follows.
The font \verb|CMEX10| contains large symbols for mathematics. They
overlap sometimes in the table. Since the baseline of the
\verb|CMEX10| font is at the top of the signs, Gnuplot defines a font
\verb|CMEX10-Baseline| with a different baseline if \verb|CMEX10| is
embedded (normaly by using \verb|fontfile 'cmex10.pfb'|.

\begin{longtable}{lllllllllllll}
  \toprule
  Oct& \rotatebox{90}{CMR10}& \rotatebox{90}{CMTI10}&
  \rotatebox{90}{CMTT10}& \rotatebox{90}{CMMI10}&
  \rotatebox{90}{CMU10}& \rotatebox{90}{CMSS10}&
  \rotatebox{90}{CMTEX10}& \rotatebox{90}{CMFF10}&
  \rotatebox{90}{CMSY10}& \rotatebox{90}{LASY10}&
  \rotatebox{90}{CMEX10-Baseline}& Oct\\
  \midrule
  \endhead
  \bottomrule
  \endfoot
  \row{001}\row{002}\row{003}\row{004}\row{005}\row{006}\row{007}
  \row{011}\row{012}\row{013}\row{014}\row{015}\row{016}\row{017}
  \row{021}\row{022}\row{023}\row{024}\row{025}\row{026}\row{027}
  \row{031}\row{032}\row{033}\row{034}\row{035}\row{036}\row{037}
  \row{041}\row{042}\row{043}\row{044}\row{045}\row{046}\row{047}
  \row{051}\row{052}\row{053}\row{054}\row{055}\row{056}\row{057}
  \row{061}\row{062}\row{063}\row{064}\row{065}\row{066}\row{067}
  \row{071}\row{072}\row{073}\row{074}\row{075}\row{076}\row{077}
  %
  \row{101}\row{102}\row{103}\row{104}\row{105}\row{106}\row{107}
  \row{111}\row{112}\row{113}\row{114}\row{115}\row{116}\row{117}
  \row{121}\row{122}\row{123}\row{124}\row{125}\row{126}\row{127}
  \row{131}\row{132}\row{133}\row{134}\row{135}\row{136}\row{137}
  \row{141}\row{142}\row{143}\row{144}\row{145}\row{146}\row{147}
  \row{151}\row{152}\row{153}\row{154}\row{155}\row{156}\row{157}
  \row{161}\row{162}\row{163}\row{164}\row{165}\row{166}\row{167}
  \row{171}\row{172}\row{173}\row{174}\row{175}\row{176}\row{177}
  %
\end{longtable}

\end{document}
