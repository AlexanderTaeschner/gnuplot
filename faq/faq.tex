%% Don't forget to change the paper format in the next line
%
%\documentclass[a4paper,11pt]{article}
\documentclass[letter,11pt]{article}
\usepackage[margin=2cm]{geometry}
\usepackage[T1]{fontenc}
\usepackage[hyphens]{url}
\urlstyle{sf}

\parindent=0pt
\parskip=5pt

\ifx\pdfoutput\undefined
    % latex or latex2html output
    \usepackage{times,mathptmx}
    \usepackage[
        hypertex,
        hyperindex,
        bookmarks,
        bookmarksnumbered=true,
        pdftitle={gnuplot faq},
        pdfauthor={gnuplot},
        pdfsubject={see www.gnuplot.info}
        % , pdfcreator={}
        % , pdfkeywords={...}
    ]{hyperref}
\else % *** pdflatex output
    \usepackage{times,mathptmx}
    \usepackage[
%       pdftex,
        hyperindex,
        bookmarks,
        bookmarksnumbered=true,
        pdftitle={gnuplot faq},
        pdfauthor={gnuplot},
        pdfsubject={see www.gnuplot.info}
        % , pdfcreator={}
        % , pdfkeywords={...}
    ]{hyperref}
\fi

\usepackage{color}
\definecolor{darkblue}{rgb}{0,0,0.5}
\hypersetup{
  colorlinks   = true, %Colours links instead of ugly boxes
  linkcolor    = darkblue, %Colour of internal links
  urlcolor     = blue %Colour for external hyperlinks
}

% There may be incompatibilities between different versions of
% url.sty, html.sty and hyperref.sty -- it seems there are machines which
% cannot combine them together with simultaneous output to dvi, pdf, html.
% Thus do it this way:
\ifx\html\undefined
    % Modified Dec 2014 (EAM) for use with htlatex or pdflatex
    \def\http#1{{\small\href{http://#1}{\url{http://#1}}}}
    \def\mailto#1{{\small\href{mailto://#1}{\url{mailto://#1}}}}
    \def\news#1{\href{news://#1}{\url{news://#1}}}
    \def\ftp#1#2{\href{ftp://#1#2}{\url{#1} in \url{#2}}}
\else
    % Running this file by latex2html:
    \usepackage{html}
%    \html{
        \newcommand{\news}[1]%
            {\def~{\~{}}\htmladdnormallink{\latex{\url{#1}}\html{\textit{#1}}}%
                {news:#1}%
            }
        \newcommand{\ftp}[2]%
            {\htmladdnormallink{\latex{\url{#1}{} in \url{#2}}%
                    \html{\textit{#1} in \textit{#2}}}%
                {ftp://#1#2}%
            }
        \newcommand{\mailto}[1]%
            {\htmladdnormallink{\latex{\url{<#1>}}\html{\textit{#1}}}%
                {mailto:#1}%
            }
        \newcommand{\http}[1]%
            {\htmladdnormallink{\latex{\url{http://#1}}%
                    \html{\textit{http://#1}}}%
                {http://#1}%
            }
%   }
\fi


% History:
% version 1.4 dated 99/10/07
% Revision 1.58 dated 2017/06/04 (last CVS version)
%

%
\newcommand{\gnuplot}{\textbf{gnuplot }}
\newcommand{\Gnuplot}{\textbf{Gnuplot }}

\begin{document}

\title{\gnuplot FAQ}
\author{}
\date{}
\maketitle

\noindent
This material describes \gnuplot version 5 (up to 5.4)
        \\
FAQ version: February 2023
        \\
PDF version of current document: {\http{www.gnuplot.info/faq/faq.pdf}.


\tableofcontents

\newpage
\setcounter{section}{-1}
\section{Meta -- Questions}

\subsection{Where do I get this document?}

The newest version of this document is on the web at
\http{www.gnuplot.info/faq/}.

\subsection{Where do I send comments about this document?}

Send comments, suggestions etc to the developer mailing list
\mailto{gnuplot-beta@lists.sourceforge.net}.

\section{General Information}

\subsection{What is \gnuplot?}

\gnuplot is a command-driven plotting program.
It can be used interactively to plot functions and data points in
both two- and three-dimensional plots in many different styles and
many different output formats.  \Gnuplot can also be used as a
scripting language to automate generation of plots.
It is designed primarily for the visual display of scientific data.
\gnuplot is copyrighted, but freely distributable;
you don't have to pay for it. You are welcome to download the source code.


\subsection{How did it come about and why is it called \gnuplot?}

The authors of \gnuplot are:
Thomas Williams, Colin Kelley, Russell Lang, Dave Kotz, John
Campbell, Gershon Elber, Alexander Woo and many others.

The following quote comes from Thomas Williams:
\begin{quote}
     I was taking a differential equation class and Colin was taking
     Electromagnetics, we both thought it'd be helpful to visualize the
     mathematics behind them. We were both working as sys admin for an
     EE VLSI lab, so we had the graphics terminals and the time to do
     some coding. The posting was better received than we expected, and
     prompted us to add some, albeit lame, support for file data.

     Any reference to GNUplot is incorrect. The real name of the program
     is "gnuplot". You see people use "Gnuplot" quite a bit because many
     of us have an aversion to starting a sentence with a lower case
     letter, even in the case of proper nouns and titles. gnuplot is not
     related to the GNU project or the FSF in any but the most
     peripheral sense. Our software was designed completely
     independently and the name "gnuplot" was actually a compromise. I
     wanted to call it "llamaplot" and Colin wanted to call it "nplot."
     We agreed that "newplot" was acceptable but, we then discovered
     that there was an absolutely ghastly pascal program of that name
     that the Computer Science Dept.\ occasionally used. I decided that
     "gnuplot" would make a nice pun and after a fashion Colin agreed.
\end{quote}


\subsection{What does \gnuplot offer?}

\begin{itemize}
\item Two-dimensional functions and data plots combining many different
 elements such as points, lines, error bars, filled shapes, labels, arrows, ...
\item Polar axes, log-scaled axes, general nonlinear axis mapping, parametric coordinates
\item Data representations such as heat maps, beeswarm plots, violin plots, histograms, ...
\item Three-dimensional plots of data points, lines, and surfaces in
many different styles (contour plot, mesh)
\item Algebraic computation using integer, floating point, or complex arithmetic
\item Data-driven model fitting using Marquardt-Levenberg minimization
\item Support for a large number of operating systems, graphics
 file formats and output devices
\item Extensive on-line help
\item \TeX{}-like text formatting for labels, titles, axes, data points
\item Interactive command line editing and history
\end{itemize}


\subsection{Is \gnuplot suitable for scripting?}

Yes. Gnuplot can read in files containing additional commands during
an interactive session, or it can be run in batch mode by piping a
pre-existing file or a stream of commands to stdin. Gnuplot is used
as a back-end graphics driver by higher-level mathematical packages
such as Octave and can easily be wrapped in a cgi script for use as a
web-driven plot generator.  Gnuplot supports context- or data-driven
flow control and iteration using familiar statements
{\em if else continue break while for}.


\subsection{Can I run \gnuplot on my computer?}

\Gnuplot{} is in widespread use on many platforms, including
MS Windows, linux, unix, and macOS.  The current source code retains
supports for older systems as well, including VMS, Ultrix, OS/2, and
MS-DOS. However 16-bit platforms are no longer supported.

You should be able to compile the \gnuplot source more or
less out of the box in any reasonably standard (ANSI/ISO C, POSIX)
environment.


\subsection{Legalities}

\Gnuplot{} is authored by a collection of volunteers, who cannot
make any legal statement about the compliance or non-compliance of
\gnuplot or its uses. There is no warranty whatsoever. Use at your own risk.

\Gnuplot{} is freeware in the sense that you don't have to pay for it.
You can use or modify gnuplot as you like, however certain restrictions
apply to further distribution of modified versions.
Please read and abide by the modification and redistribution terms in
the \textit{Copyright} file.  Some individual source files are explicitly
dual-licensed;  in those cases alternative terms for redistribution of code
in that specific file are listed at the head of the file.

\subsection{Does \gnuplot have anything to do with the FSF and the GNU
project?}

\Gnuplot{} is neither written nor maintained by the FSF. At one time it
was distributed by the FSF but this is no longer true. \Gnuplot{} as a whole
is not covered by the GNU General Public License (GPL).


\subsection{Where do I get further information?}

See the main gnuplot web page \http{www.gnuplot.info}.

Some documentation and tutorials are available in other languages than English.
See \http{gnuplot.sourceforge.net/help.html}, section "Localized learning pages
about gnuplot", for the most up-to-date list.


\section{Setting it up}

\subsection{What is the current version of \gnuplot?}

The current stable version of \gnuplot is 5.4, first released in July 2020.
Incremental versions (patchlevel 1, 2, ...) are typically released every six months.
The development version is currently \gnuplot 6.

\subsection{Where can I get \gnuplot?}
\label{where-get-gnuplot}

The best place to start is \http{www.gnuplot.info}. From there
you find various pointers to other sites, including the project
development site on SourceForge \http{sourceforge.net/projects/gnuplot}.

The source distribution ("gnuplot-5.4.0.tar.gz" or a similar name) is
available from the official distribution site \http{sourceforge.net/projects/gnuplot}.

\subsection{Why would I care about the development version?}

The current development version will generally include features that are
not yet part of the most recent stable release of gnuplot.
%As of July 2020 the 5.5 development version notably supports
%\begin{itemize}{}{}
%\setlength\itemsep{-4pt}
%\item named colormaps to supplement the primary palette
%\item along-path smoothing options for 2D and 3D curves
%\end{itemize}

\subsection{Where can I get current development version of \gnuplot?}

The development version of gnuplot is held in a git repository which you
can clone as shown below to inspect or build an executable program
from the source.

\scriptsize
\begin{verbatim}
  git clone https://git.code.sf.net/p/gnuplot/gnuplot-main gnuplot
\end{verbatim}
\normalsize

Questions related to the development version should go to
\mailto{gnuplot-beta@lists.sourceforge.net}.


\subsection{How do I compile \gnuplot on my system?}

Read the release notes and files \textit{README} and \textit{INSTALL}.
You will need C and C++ compilers and installed versions of various
support libraries depending on what configuration options you choose and what
terminal types you want your executable to support.

\begin{itemize}
\item
To build from a release version on linux, use \textit{./configure} (or \textit{./configure {-}{-}prefix=\$HOME/usr}
for an installation for a single user), \textit{make} and finally
\textit{make install}.  Pay close attention to the output from the \textit{configure}
script (yes there is a lot of it).
There you will find hints as to what additional support libraries may be needed and
what additional options may be desirable.  In general you will need to have installed
the "development" package for each of the support libraries.
\item
To build from the development source on linux you must run a separate script \textit{./prepare}
prior to running \textit{./configure} .
\item
On Windows, makefiles can be found in \textit{config/mingw}, \textit{config/msvc},
\textit{config/watcom}, and \textit{config/cygwin}. Update the options in the
makefile's header and run the appropriate \textit{make} tool in the same directory
as the makefile. Additional instructions can be found in the makefiles.
\item
For other platforms, copy the relevant makefile (e.g. \textit{makefile.os2} for
OS/2) from \textit{config/} to \textit{src/}, optionally update options in the
makefile's header, then change directory to \textit{src} and run \textit{make}.
\end{itemize}


\subsection{What documentation is there, and how do I get it?}

Full documentation is included in the release distribution as a PDF file.
Individual sections can be browsed from inside a gnuplot session
by typing \textit{help {\em keyword}}.
Documentation in other formats can be compiled from source in the
{\em docs} subdirectory.

Online copies in English and Japanese are available at
\http{gnuplot.sourceforge.net/documentation.html}.

\subsection{Worked examples}

There is a directory of worked examples in the the source distribution.
Many of these examples and the resulting plots may also be found online at
\http{gnuplot.sourceforge.net/demo/}.


\subsection{How do I determine which options were compiled into my \gnuplot executable?}

Given that you have a compiled version of \gnuplot, you can use the
{\em show} command to display a list of configuration and build options
that were used to build your copy.  The output formats (a.k.a. "terminals")
built into your copy of gnuplot are reported by {\em set terminal}.

\small
\begin{verbatim}
gnuplot> show version long
gnuplot> set terminal
\end{verbatim}
\normalsize


\section{Working with it.}

\subsection{How do I get help?}

Give the {\em help} command at the initial prompt. After that, keep
looking through the keywords. Good starting points are {\em help plot}
and {\em help set}.

\begin{itemize}
\item
Read the manual, if you have it.

\item
Run the demos in the {\em demo} subdirectory or look at the
online copies. They should give you some ideas.
\http{gnuplot.info/demo}

\item
Ask your colleagues, the system administrator, or the person who
set up \gnuplot.

\item
There is an old-school usenet group devoted to gnuplot questions
\news{comp.graphics.apps.gnuplot}
inhabited mostly by users who found it last century.

\item
A more active help forum may be found on StackOverflow
\http{stackoverflow.com/questions/tagged/gnuplot}

\item
If all these fail, send mail to the gatewayed mailing list
\mailto{gnuplot-info@lists.sourceforge.net}.
Please note that to reduce the amount of spam it would otherwise receive,
you must subscribe before you can post to it. Subscription instructions
may be found at
\http{lists.sourceforge.net/lists/listinfo/gnuplot-info}.

\end{itemize}
When asking a question, always mention the \gnuplot version and
operating system you are using.  If you are asking about a plot that
did not come out the way you expected, please try to show a minimal
set gnuplot commands that produce a plot showing the problem.


\subsection{How do I print out my graphs?}

The output format produced by a plot command is determined by a
prior {\em set terminal} command.  For non-interactive output
you should pair this with a {\em set output} command to provide
a file name.

As an example, the following session first plots a graph of sin(x) to the
screen and then redraws that same plot as a PDF output file.
Note: the PDF plot may not look exactly like the plot on the screen.

\small
\begin{verbatim}
gnuplot> plot sin(x)
gnuplot> set terminal pdf
Terminal type is now 'pdfcairo'
Options are ' transparent enhanced fontscale 0.5 size 5.00in, 3.00in '
gnuplot> set output "sin.pdf"
gnuplot> replot
gnuplot> unset output            # close output file (otherwise it stays open)
gnuplot> unset terminal          # return to default interactive terminal
gnuplot>
\end{verbatim}
\normalsize

If the start point is not the default interactive terminal you
can accomplish the same thing using {\em push} and {\em pop}
\small
\begin{verbatim}
gnuplot> set terminal push    # save current terminal type (may not be default)
gnuplot> set terminal pdf
gnuplot> set out 'a.pdf'
gnuplot> replot
gnuplot> unset out
gnuplot> set term pop         # restore saved terminal type
\end{verbatim}
\normalsize

Some interactive terminal types ({\em win, wxt, qt}) provide a printer icon
on their toolbar. This widget prints the current plot or saves it to file
using generic system tools rather than by using a different gnuplot terminal type.
That is, the file you get by selecting "save to png" in the print menu may be
different than the file you get from {\em set term png; replot;}.
In general a plot saved in this way will more closely reproduce the screen image than
a plot generated from the command line by changing the terminal type.


\subsection{How do I include my graphs in my favorite word processor?}

Basically, you save your plot to a file in a format your word processor
can understand, and then you read in the file from your word processor. Vector
formats (PostScript, emf, svg, pdf, \TeX{}, \LaTeX{}) should be preferred,
as you can scale your graph later to the right size.

Use {\em set term} to get a list of available file formats.

Many word processors can use Encapsulated PostScript (*.eps) for graphs.
You can generate eps output in \gnuplot using either
{\em set terminal postscript eps}
or
{\em set terminal epscairo}
.
\Gnuplot does not embed a bitmap preview image in the output eps file.
To accommodate some word processors you may have to add a preview image yourself
using an external tool before importing it into the word processor.

Some applications, including the LibreOffice and Microsoft Office suites,
can handle vector images in EMF format. These can be either produced by the emf
terminal, or by selecting 'Save as EMF...' from the toolbar of the
windows terminal plot window.

LibreOffice can also read SVG, as well as AutoCAD's dxf format.

There are many ways to use gnuplot to produce graphs for inclusion in a
\TeX\ or \LaTeX\ document.
Some terminals produce *.tex fragments for direct inclusion; others
produce *.eps, *.pdf, *.png output to be included using the
\textbackslash{}includegraphics command.
The epslatex and cairolatex terminals produce both a graphics
file (*.eps or *.pdf) and a *.tex document file that refers to it.
The tikz terminal produces full text and graphics to a pdf file
when the output is processed with pdflatex.

Most word processors can import bitmap images (png, pbm, etc).
The disadvantage of this approach is that the resolution of your
plot is limited by the size of the plot at the time it is generated
by \gnuplot, which is generally a much lower resolution than the
document will eventually be printed in.


\subsection{How do I edit or post-process a \gnuplot graph?}

This depends on the terminal type you use.

\begin{itemize}

\item \textbf{svg} terminal (scalable vector graphics) output can
be further edited by a svg editor, e.g.
\textbf{Inkscape} (\http{www.inkscape.org}),
\textbf{Skencil} (\http{www.skencil.org}) or
\textbf{Dia} (\http{projects.gnome.org/dia/}), or loaded
into \textbf{OpenOffice.org} with an on-fly conversion into OO.o Draw
primitives.

\item PostScript or PDF output can be edited directly by tools such
as Adobe Illustrator or Acrobat, or can be converted to a variety
of other editable vector formats by the \textbf{pstoedit} package.
Pstoedit is available at \http{www.pstoedit.net}.

\item The DXF format is the AutoCAD's format, editable by several
other applications.

\item Bitmapped graphics (e.g. png, jpeg, pbm) can be edited using
tools such as ImageMagick or Gimp.
In general, you should use a vector graphics program to post-process
vector graphic formats, and a pixel-based editing program
to post-process pixel graphics.

\end{itemize}

\subsection{How do I save and restore my current settings?}

Use the {\em save "filename"} and {\em load "filename"} commands.

\subsection{Can I put both commands and data into a single file?}

\Gnuplot version 5 supports named blocks of data in "here document" format:
\small
\begin{verbatim}
gnuplot> $DATABLOCK << EOD
  cats 4 2
  dogs 1 4
EOD
gnuplot> plot $DATABLOCK using 2:3:1 with labels
\end{verbatim}
\normalsize
Once the named block has been defined, it can be used as many times
as you like.

Data can also be provided in-line as part of a plot command using the
pseudo-file "-".  In this case the data can only be used once.

\small
\begin{verbatim}
gnuplot> plot "-"
1 1
2 4
3 9
e
\end{verbatim}
\normalsize

\subsection{How do I run my data through a filter before plotting?}

If your operating system supports the popen() function, you
can filter input data through another program or system utility
as part of the {\em plot} command.

\small
\begin{verbatim}
gnuplot> plot "< sort +2 file.in"  # pre-sort data on column 2
\end{verbatim}
\normalsize

This mechanism is particularly powerful in combination with the
unix-derived command line utilities {\em awk}, {\em sort} and {\em grep}.

\subsection{Can I use \gnuplot routines for my own programs?}

On systems supporting pipes, you can pipe commands to \gnuplot from other
programs. Many applications with gnuplot as the graphics engine, like Octave
(\http{www.octave.org}), use this method. This also works from a cgi script to
drive \gnuplot from a forms-based web page.


\section{Customizing the appearance of your plot}

\subsection{How to inspect or change the default colors, line, and point properties?}

When you issue a {\em plot} or {\em splot} command with multiple components,
\gnuplot will by default cycle through a set of colors and linetypes.
You can override this by providing specific color or linetype properties in the
plot command or you can change the default sequence.
Each of the commands below accepts many additional parameters
\begin{list}{}{}
  \item
  {\em test} displays the active colors, line and point properties, etc
             for the current terminal type
  \item
  {\em set color} or {\em set monochrome} selects a pre-defined sequence.
  \item
  {\em set linetype} changes the properties of an existing linetype or adds a new one.
  \item
  {\em set palette} changes the color palette used for pm3d modes such as heat maps.
  \item
  {\em set pointsize} scales all points by an additional factor
\end{list}

\subsection{Hidden line or surface removal}

There are two relevant commands. {\em set hidden3d} affects surfaces
drawn using the 3D plot style {\em splot ... with lines}. It also clips
line segments created by other 3D plot styles that are occluded by those
surfaces.   However it does not handle plots generated in {\em pm3d} mode.
This includes styles {\em with pm3d}, {\em with zerror}, {\em with boxes},
and miscellaneous plot elements generated while {\em set pm3d} is in effect.
Hidden surface removal for these plots is achieved instead by drawing them
in order of their distance from the viewer, enabled by the command
{\em set pm3d depthorder}.

\subsection{How do I force exact positions for the graph borders on the page?}

Specify the position of the top, bottom, left, and right borders in
terms of their fractional position within the page:

\small
\begin{verbatim}
set lmargin at screen 0.05
set bmargin at screen 0.05
set rmargin at screen 0.95
set tmargin at screen 0.95
\end{verbatim}
\normalsize

\subsection{Arranging multiple plots next to each other on a single page}

The command you want is {\em set multiplot}.  The program can place
a specified number of plots on a regular grid
({\em set mulplot layout <rows>, <columns> ...}) or you can position them
one by one using {\em set origin} and {\em set size}.

\subsection{How to request 1:1 scaling of axes?}

Try {\em set size square} or {\em set view equal xy}.

\subsection{Palette that works for both color and black\&white printing?}

Try {\em set palette cubehelix}.

\subsection{How do I skip data points?}

By specifying \textit{?} as a data value, as in
\small
\begin{verbatim}
        1 2
        2 3
        3 ?
        4 5
\end{verbatim}
\normalsize

See also \textit{set missing}.
See also \textit{set datafile commentschars} for specifying comment characters in
data files.


\subsection{How do I plot every nth point?}

This can be specified with various options for the command {\em plot},
for example {\em plot 'a.dat' every 2}.  If you want to draw a line
through every point but only draw a point symbol at every nth point,
then try {\em plot 'a.dat' with linespoints pointinterval n}.


\subsection{How do I plot a vertical line?}

Depending on context, the main methods are:
\begin{itemize}
\item {\em set arrow .... .... nohead} where you have to compute
explicitly the start and the end of the arrow.
\item generate inlined datapoints and plot them
\end{itemize}

\subsection{Y axis label in the wrong place and/or left margin too big}
\Gnuplot has trouble estimating how much space to the left of the plot
will be required to hold the Y axis label if it is printed horizontally
({\em set ylabel norotate}).  This is particularly true if there is
TeX markup in the label string.  To work around this problem you can place
the text in a numbered label rather than in {\em ylabel}.
\small
\begin{verbatim}
Y = 1001
set label Y '$\operatorname{\mathfrak{Im}} S_{21}$'
set label Y norotate at graph 0.0, 0.5 offset -6
\end{verbatim}
\normalsize

\section{Plot types that people often ask about}

\subsection{Animations}

\Gnuplot versions through release 5.4 support only one terminal type
(gif) that directly outputs an animated file:
\begin{verbatim}
set terminal gif animate {delay <time>} {loop <N>} {optimize}
\end{verbatim}

The development version also supports animation in webp format.
Have a look at
\http{gnuplot.sourceforge.net/demo/animate2.html}
in the demo collection.

\subsection{Implicit defined graphs}

Implicit graphs or curves cannot be plotted directly in \gnuplot.
However there is a workaround.
\small
\begin{verbatim}
gnuplot> # An example. Place your definition in the following line:
gnuplot> f(x,y) = y - x**2 / tan(y)
gnuplot> set contour base
gnuplot> set cntrparam levels discrete 0.0
gnuplot> unset surface
gnuplot> set table $TEMP
gnuplot> splot f(x,y)
gnuplot> unset table
gnuplot> plot $TEMP w l
\end{verbatim}
\normalsize
The trick is to draw the single contour line z=0 of the surface
z=f(x,y), and store the resulting contour curve to a temporary file or datablock.


\subsection{How to fill an area between two functions}

A plot with filled area between two functions f(x) and g(x) can be obtained using
the pseudo file '+' with  plot style {\em filledcurves}.
\small
\begin{verbatim}
f(x)=cos(x); g(x)=sin(x)
set xrange [0:pi]
plot '+' using 1:(f($1)):(g($1)) with filledcurves
\end{verbatim}
\normalsize

Note that this code fragment fills area between the two curves regardless of
which one is above the other.  If you want to fill only the area satisfying g(x)<f(x)
or f(x)<g(x)
add an additional keyword {\em above} or {\em below} after {\em filledcurves}.



\subsection{Drawing a 2D projection of 3D data}

The command {\em set view map} adjusts the view angle and scaling such
that subsequent 3D graphs made with {\em splot} have approximately the same layout
as a 2D graph made with {\em plot}, with the x axis horizontal and the y axis vertical.
Version 5.4 commands {\em set view projection xz} and {\em set view projection yz}
similarly initialize layouts for 2D projection of the xz or yz plane, with the
z axis horizontal and the x or y axis vertical.

\subsection{How to overlay dots/points scatter plot onto a pm3d map/surface}

Use the {\em explicit} option of the pm3d style:
\small
\begin{verbatim}
gnuplot> set pm3d explicit
gnuplot> splot x with pm3d, x*y with points
\end{verbatim}
\normalsize


\subsection{How to produce labeled contours}

Labeling individual contours in a contour plot used to require special
tricks and extra processing steps in \gnuplot version 4.
Since version 5 it is much simpler.  Plot the contours twice,
once "with lines" and once "with labels".  To make the labels stand out
it may help to use
\begin{verbatim}
set style textbox opaque noborder
set contours
splot 'DATA' with lines, 'DATA' with labels boxed
\end{verbatim}

\subsection{Does \gnuplot support bar-charts/histograms/boxes?}

\Gnuplot{} supports various clustered and stacked histogram styles to display
pre-tabulated data.  It also offers a few options for accumulating raw data
into bins, which can in turn be displayed as a bar chart.  See the documentation
for \textit{bins} and for \textit{smooth frequency}.

\subsection{Does \gnuplot support pie charts? quarterly time charts?}

Pie charts are sort of difficult in \gnuplot, but see
\http{gnuplot.sourceforge.net/demo/circles.html},
or have a look at
\http{gnuplot-tricks.blogspot.com/2009/08/pie-charts-entirely-in-gnuplot.html}.

The demo collection contains an example of a simple Gantt chart.

\subsection{Does \gnuplot support multiple y-axes on a single plot?}

Yes. 2D plots can have separate x axes at the bottom (x1) and top (x2),
and separate y axes at the left (y1) and right (y2).  Version 5 offers a
plot mode \textit{with parallelaxes} that allows any number of additional
y axes to be defined.

\subsection{How to draw a solid made of triangular facets?}

Version 5.4: plot style \textit{with polygons} can handle polygonal faces
(triangles, rectangles, octagons, ...) in either 2D or 3D plots.

In older versions the best you can do is to describe colored quadrangles
that are facets of a 3D object using a file organized like this:
\small
\begin{verbatim}
# triangle 1
x0 y0 z0 <c0>
x1 y1 z1 <c1>

x2 y2 z2 <c2>
x2 y2 z2 <c2>


# triangle 2
x y z
...
\end{verbatim}
\normalsize

Notice the single and double blank lines. Also notice that each triangle
really has four vertices of which two are identical.  This is because the
pm3d code only knows how to deal with quadrangles.
\textit{<cN>} is an optional color.

Then plot it:
\small
\begin{verbatim}
set pm3d
set style data pm3d
set pm3d depthorder
splot 'facets.dat' using 1:2:3 title "default coloring"
splot 'facets_with_color.dat' using 1:2:3:4 title "explicit colors"
\end{verbatim}
\normalsize

Gnuplot is not a 3D modeling program.
The depthorder rendering in version 5 does not handle inter-penetrating facets;
version 6 does somewhat better.
For true 3D rendering you would be probably be better off using a ray-tracing program.

\subsection{How do I plot two functions in non-overlapping regions?}

Give the desired range immediately before each function being plotted.
For example, to plot experimental data and two different functional
models f1 and f2 covering two different portions of the domain:
\small
\begin{verbatim}
gnuplot> set autoscale x  # get x range from the data
gnuplot> plot 'data', [*:0] f1(x), [0:*] f2(x)
\end{verbatim}
\normalsize


\subsection{How do I plot lines (not grids) using splot?}

If the data input to splot is arranged such that each line contains
the same number of data points (using blank lines as delimiters, as usual),
splot will by default treat the data as describing a surface.
If you want to draw individual lines instead, try some combination of
\textit{unset surface}, \textit{set surface explicit}, \textit{plot ... nosurface}.


\subsection{How do I plot a function f(x,y) that is bounded by other
          functions in the x-y plane?}

Here is one way:
\small
\begin{verbatim}
gnuplot> f(x,y) = x**2 + y **2
gnuplot> x(u) = 3*u
gnuplot> yu(x) = x**2
gnuplot> yl(x) = -x**2
gnuplot> set parametric
gnuplot> set cont
gnuplot> splot [0:1] [0:1] u,yl(x(u))+(yu(x(u)) - yl(x(u)))*v,\
       >                   f(x(u), (yu(x(u)) - yl(x(u)))*v)
\end{verbatim}
\normalsize


\section{Text formatting and special symbols}

\subsection{Text markup using "enhanced text" mode}

Starting with version 5, \gnuplot defaults to "enhanced text" mode,
in which text markup is indicated by a set of special characters embedded in the text.

\begin{center}
\begin{tabular}{|clll|} \hline
\multicolumn{4}{|c|}{Enhanced Text Control Codes} \\ \hline
Control & Example & Result & Explanation \\ \hline
\verb~^~ & \verb~a^x~ & $a^x$ & superscript\\
\verb~_~ & \verb~a_x~ & $a_x$ & subscript\\
\verb~@~ & \verb~a@^b_{cd}~ & $a^b_{cd}$ &phantom box (occupies no width)\\
\verb~&~ & \verb~d&{space}b~ & d\verb*+     +b & inserts space of specified length\\
\verb|~| & \verb|~a{.8-}| & $\tilde{a}$ & overprints '-' on 'a', raised by .8\\
\verb~ ~ & \verb~ ~ & ~ ~ & times the current fontsize\\
\verb| | & \verb|{/Times abc}| & {\rm abc} & print abc in font Times at current size\\
\verb| | & \verb|{/Times*2 abc}| & \Large{\rm abc} & print abc in font Times at twice current size\\
\verb| | & \verb|{/Times:Italic abc}| & {\it abc} & print abc in font Times with style italic\\
\verb| | & \verb|{/Arial:Bold=20 abc}| & \Large\textsf{\textbf{abc}} & print abc in boldface Arial font size 20\\
%% 5.3 only %% \verb|\U+| & \verb|\U+221E| & $\infty$ & Unicode point U+221E INFINITY\\
\hline
\end{tabular}
\end{center}

\subsection{How do I turn text markup on or off?}

To exempt a particular text string from this processing mode use the keyword
{\em noenhanced}.  For example we don't want to interpret file names as subscripts:
\small
\begin{verbatim}
set title 'Compare file_1.dat and file_2.dat' noenhanced
\end{verbatim}
\normalsize

\subsection{Is UTF-8 the answer to all my special character problems?}

Yes.

Unfortunately there are some circumstances where it very cumbersome to use,
notably in generating PostScript output.
If you are working in a UTF-8 computing environment, you probably do not have to
do anything special in gnuplot to use it.
If you are not, then you can still tell gnuplot to use UTF-8 for output:
{\em set encoding utf8}.

If you cannot enter UTF-8 characters on your keyboard you will have to solve
that outside of gnuplot. Or you can use octal escape sequences to type them in
byte-by-byte, or (since version 5.4) unicode escape sequences like
{\em \verb&\U+221E&} ($\infty$).
If your keyboard does generate UTF-8 but you don't know what keystrokes would
produce a specific character, there are probably suitable character selection
apps for your desktop (e.g. KDE {\em kcharselect} or GNOME {\em Character Map}).

\subsection{What if I need h-bar (Planck's constant)?}

The most straightforward way is to use a UTF-8 font, and type in the
$\hbar$ character (Unicode {\verb&\U+210F&}) directly.

PostScript: PostScript does not handle utf8 easily so you must use an
approximation based on enhanced text markup and possibly a special
Symbol font:

\verb+ @{/=56 -} {/=24 h}+ or
\verb+ {/=8 @{/Symbol=24 -} _{/=14 h}}+
In the latter, the "-" (a long one in /Symbol) is non-spacing and 24-pt.
The 14-pt "h" is offset by an 8-pt space (which is the space preceding
the "\_") but smaller, since it's written as a subscript.
But these don't look too much like the hbar we're used to, since the bar
is horizontal instead of sloped.
Another possibility is \verb+{/=14 @^{/Symbol=10 -}{/=14 h}}+.

The reduced Planck's constant can be set very easily by using the
AMS-LaTeX PostScript fonts which are available from
\http{www.ams.org/tex/amsfonts.html} (also included in many LaTeX
distributions). \Gnuplot{} (see \verb+help fontpath+) and the
PostScript interpreter (usually Ghostscript) have to know where the
file \verb+ msbm10.pfb+ (or \verb+ msbm10.pfa+) resides. Use
\verb+ {/MSBM10 \175}+ to produce \verb+ \hslash+ which is a "h"
superimposed by a sloped bar. The standard \verb+ \hbar+ (horizontal
bar) has the octal code 176. Please note that h-bar exists only as an
italic type.

\subsection{What if I need the Solar mass symbol?}

As with Planck's constant, the most straightforward way is to use a
UTF-8 font, and type in the $\odot$ character
(Unicode {\verb&\U+2299&} ; "circled dot operator")
directly.  The very similar glyph at (Unicode code point {\verb&\U+2609&} ; "sun")
may be even better, but not many fonts provide it.

\subsection{How to use Greek letters or other special symbols?}

The old-style way is to use enhanced text mode to switch to a specialize font,
e.g. the Adobe "Symbol" font, that maps the characters you want onto ascii
letters 'a', 'b', etc.  This may still be necessary for PostScript output.
However a much simpler way is to select UTF-8 encoding and enter the
special characters just as you would any other text.
This obviates the need to change fonts and gives you access to all unicode
code points including CJK character sets.  To actually print or view the
files produced by \gnuplot you still need appropriate fonts installed on
your computer or output device.  \Gnuplot itself does not provide fonts.

The various \LaTeX{} terminal types (\textit{latex, epslatex, tikz, context, cairolatex})
hand off text generation to \LaTeX{}. In this case you can use normal \LaTeX{}
markup like \verb+$\alpha_{3}$+ ($\alpha_{3}$).

\subsection{How do I include accented characters?}

If you are stuck using a non-utf8 encoding you should use
8-bit character codes together with the appropriate encoding option
to obtain accented characters like \"u or \^n in your labels.
You can represent the 8-bit code with an escape sequence.
See the following example:

\small
\begin{verbatim}
gnuplot> set encoding iso_8859_1
gnuplot> set title "M\374nchner Bierverbrauch \374ber die Jahre"
gnuplot> plot "bier.dat" u 1:2
\end{verbatim}
\normalsize

Everyone else should use UTF-8, where these are normal characters.

\subsection{Can I put different text sizes into my plots?}

Most terminal types allow you to specify a starting font face and size.
The "enhanced text" mode allows you to change fonts, text sizes, bold and
italic styles within a plot.



\section{Miscellaneous}

\subsection{Does \gnuplot support a driver for graphics format XXX?}

To see a list of the available graphic drivers for your installation of
\gnuplot, type {\em set term}.

Some graphics drivers are included in the source distribution
but are not built by default. If you want to use them, you'll
have to recompile from source.

\subsection{I've found a bug, what do I do?}

First, try to see whether it actually is a bug, or whether it
is a feature you can turn off with some obscure {\em set} command.

Next, see whether you have an old version of \gnuplot; if so,
chances are the bug has been fixed in a newer release.

Before submitting a bug report, please check whether the bug in question
has already been fixed in the upstream source since the release of the
current version. These are marked "pending-fixed" on the bug tracker.

If, after checking these things, you still are convinced that there is a
bug please report it using the bug-tracker at
\http{sourceforge.net/p/gnuplot/bugs}.
Be sure to include the version of \gnuplot (including patchlevel) and
the operating system you are running it on.
It helps a lot if you can provide a simple script that reproduces the error.

The tracker on sourceforge is for reporting bugs and collecting bug fixes
that will appear in a subsequent release.
The various online forums are better places to ask about
work arounds or actually solving \gnuplot related problems.

\subsection{Can I do heavy-duty data processing with \gnuplot? or
What is beyond \gnuplot?}

\gnuplot by itself is not very well suited for heavy numerical computation.
On the other hand it can easily handle very large data sets (millions of points).
If you have a specific application where the limitation is gnuplot's speed in
evaluating a non-trivial function, it might be worth coding this function in
C or C++ and making it a loadable plugin that gnuplot can call.

Programs you might look into if you need intensive numerical computation:

\textbf{octave} (\http{www.octave.org})
is a high-level language primarily intended for numerical computations.
Octave is licensed under GPL, and in principle, it is a free Matlab clone.
It provides a convenient command line interface for solving linear and nonlinear problems
numerically.  By the way, octave uses \gnuplot as its plotting
engine, so you get a data-processing program on top of \gnuplot.

\textbf{scilab} (\http{www.scilab.org}) is another open source alternative to \textbf{matlab}.

\textbf{julia + Gaston} (\http{github.com/mbaz/Gaston.jl})
The Julia language is designed for numerical analysis and computational science.
Gaston is a julia package that provides an interface to gnuplot for graphical output.

\subsection{What if I need special function that gnuplot doesn't have?}

\Gnuplot version 6 provides a greatly expanded set of complex special functions.
If the one you need is not included, you may have to use another program or write
a custom plugin for gnuplot.


\subsection{How to use hotkeys in my interactive terminals}

Most of the interactive terminals support both pre-defined and user-defined
hotkeys to replot, toggle plot elements, change axis scaling, and so on.
Hit \textit{h} in an active gnuplot plot window to get list of hotkeys.
Read \textit{help mouse} and \textit{help bind} for more information.


\subsection{I want to help in developing the next version of \gnuplot.
What can I do?}

Join the \gnuplot beta test mailing list by sending a mail
containing the line
\verb+subscribe gnuplot-beta+
in the body (not the subject) of the mail to
\mailto{Majordomo@lists.sourceforge.net}.


\section{Common problems}


\subsection{Help! None of my fonts work.}

Gnuplot does not do font handling by itself; it must necessarily leave
that to the individual device support libraries. Unfortunately, this
means that different terminal types need different help in finding
fonts. Here are some quick hints. For more detailed information please
see the gnuplot documentation for the specific terminal type you are
having problems with.

\begin{description}
\item [{png/jpeg/gif}] These terminal types use the libgd support library, which
searches for fonts in the directories given in the environmental variable
GDFONTPATH. Once you get libgd fontpaths sorted out, you will probably
want to set a default font for gnuplot.
For example: \verb+setenv GNUPLOT_DEFAULT_GDFONT verdana+
\item [{post}] PostScript font names are not resolved until the document
is printed. Gnuplot does not know what fonts are available to your
printer, so it will accept any font name you give it. However, it
is possible to bundle a font with the gnuplot output; please see the
instructions given by gnuplot's internal command {}``help set term
post fontfile''.
\item [{svg}] Font handling is viewer-dependent.
\item [{x11}] The x11 terminal uses the normal x11 font server mechanism.
The only tricky bit is that in order to use multi-byte fonts you must
explicitly say so:
\small\begin{verbatim}
set term x11 font "mbfont:sazanami mincho,vera,20"
\end{verbatim}\normalsize
\item [{win}] Select "Choose font..." from the "Options" pull-down menu
in the toolbar.
\item [{wxt, qt}] On linux systems these terminals rely on fonts provided
by the system's \textit{fontconfig} utility.
\end{description}

\subsection{The first plot in a qt terminal session fails or has bad layout}

You may also get an error message about "slow font initialization".
This is because qt relies on a shared system font cache.
If you request a font that no one else is using, it takes a while to
update the cache.  This mostly happens on Windows or macOS, because other
programs on those systems tend to use a different font mechanism so the 
relevant font cache may be empty.
Try invoking gnuplot with command line option {\em -\@-slow}.

\subsection{Pm3d splot from a datafile does not draw anything}

You do {\em set pm3d; splot 'a.dat'} and no plot but colorbox appears.
Perhaps there is no blank line in between two subsequent scans (isolines) in
the data file? Add blank lines! If you are curious what this means, then don't
hesitate to look to files like {\em demo/glass.dat} or {\em demo/triangle.dat}
in the gnuplot demo directory.

You can find useful the following awk script (call it e.g. {\em addblanks.awk})
which adds blank lines to a data file whenever number in the first column
changes:
\small
\begin{verbatim}
/^[[:blank:]]*#/ {next} # ignore comments (lines starting with #)
NF < 3 {next} # ignore lines which don't have at least 3 columns
$1 != prev {printf "\n"; prev=$1} # print blank line
{print} # print the line
\end{verbatim}
\normalsize

Then, either preprocess your data file by command
{\em awk -f addblanks.awk <a.dat}
or plot the datafile under a unixish platform by\\
{\em gnuplot > splot "<awk -f addblanks.awk a.dat"}.

\subsection{Why does \gnuplot ignore my very small numbers?}

For some purposes \Gnuplot{} treats numbers less than 1.e-08 as being zero.
Thus if you are trying to plot a collection of very small numbers,
they may be plotted as zero. Worse, if you're plotting
on a log scale, they will be off scale. Or, if the whole set of
numbers is "zero", your range may be considered empty:

\small
\begin{verbatim}
gnuplot> plot 'test1'
Warning: empty y range [4.047e-19:3e-11], adjusting to [-1:1]
gnuplot> set yrange [4e-19 : 3e-11]
gnuplot> plot 'test1'
              ^
y range is less than `zero`
\end{verbatim}
\normalsize

The solution is to change gnuplot's idea of "zero":
\small
\begin{verbatim}
gnuplot> set zero 1e-20
\end{verbatim}
\normalsize

For more information, type {\em help set zero}.

\subsection{When I {\em replot} or {\em resize} a multiplot I get only a fragment}

\Gnuplot only retains enough information to regenerate the most recent
{\em plot} or {\em splot} command.
In order to reproduce the entire multiplot you need to save the complete
sequence of commands that generated it into a script file.
Then you can {\em load} that script into \gnuplot as many times as you need
for replotting the drawing to different terminals or output files.

The same problem arises when you resize a multiplot displayed in an interactive
terminal window, because the resize event normally triggers a replot.
The {\em qt} and {\em wxt} terminals provide a toggle in the toolbox widget
that suppresses this replot on resize.  The {\em x11} terminal provides a 
terminal option {\em set term x11 noreplotonresize}.

\subsection{My formulas (like 1/3) are giving me nonsense results! What's going on?}

\Gnuplot{} does integer, and not floating point, arithmetic on
integer expressions. For example, the expression 1/3 evaluates
to zero. If you want floating point expressions, supply
trailing dots for your floating point numbers. Example:


\small
\begin{verbatim}
gnuplot> print 1/3
                0
gnuplot> print 1./3.
                0.333333
\end{verbatim}
\normalsize

This way of evaluating integer expressions is shared by both C and Fortran.


\subsection{My output files are incomplete!}

You may need to flush the output and close the file with
{\em set output} or {\em unset output}).

Some output formats (postscript, pdf, latex, ...) can include several
pages of plots in a single output file.  For these output modes, gnuplot
leaves the file open after each plot so that you can add additional plots
to it.  The file is not completed and made available to external applications
until you issue a {\em set} or {\em unset output} command,
select a different terminal type ({\em set term}), or exit gnuplot.


\subsection{Calling gnuplot in a pipe or from a script doesn't show a plot on my screen!}

One common cause is that \gnuplot exits immediately after drawing,
the plot window closes when gnuplot exits, and it all happens too fast
for you see the plot.
There are several solutions to this:
\begin{itemize}
\item  Use the gnuplot command line option {\em -persist}.
       This leaves the plot window open after gnuplot itself exits.
\item  Interactive terminals also accept {\em persist} as an option to {\em set term}.
\item  Send a {\em pause mouse close} command to gnuplot before closing the pipe.
       This will tell gnuplot not to exit until the plot window is closed by
       some other action.
\item  Have the script close the pipe only when it is sure you don't need
       gnuplot and its plot window any more.
\end{itemize}

Here is a short Perl-script showing two of these fixes:
\small
\begin{verbatim}
#!/usr/local/bin/perl -w
open (GP, "|/usr/local/bin/gnuplot -persist") or die "no gnuplot";
# force buffer to flush after each write
use FileHandle;
GP->autoflush(1);
print GP,"set term x11; plot sin(x) with lines\n";
print GP,"pause mouse close\n";
close GP
\end{verbatim}
\normalsize


\section{Credits}

This FAQ was initially compiled by John Fletcher with contributions
from Russell Lang, John Campbell, David Kotz, Rob Cunningham, Daniel
Lewart and Alex Woo. Reworked by Thomas Koenig from a draft
by Alex Woo, with corrections and additions from Alex Woo, John
Campbell, Russell Lang, David Kotz and many corrections from Daniel Lewart.
Again reworked for \gnuplot 3.7 by Alexander Mai and Juergen v.Hagen
with corrections by Lars Hecking, Hans-Bernhard Broecker and others.
Revised for \gnuplot version 4 by Petr Mikul\'{\i}k and Ethan Merritt.
Revised for \gnuplot version 5 by Ethan Merritt.

\end{document}
%%
