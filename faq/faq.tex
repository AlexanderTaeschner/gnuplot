%% Don't forget to change the paper format in the next line
%
\documentclass[a4paper,11pt]{article}
\usepackage[margin=2cm]{geometry}
\usepackage[T1]{fontenc}
\usepackage[hyphens]{url}
\urlstyle{sf}

\ifx\pdfoutput\undefined
    % latex or latex2html output
    \usepackage{times,mathptmx}
    \usepackage[
        hypertex,
        hyperindex,
        bookmarks,
        bookmarksnumbered=true,
        pdftitle={gnuplot faq},
        pdfauthor={gnuplot},
        pdfsubject={see www.gnuplot.info}
        % , pdfcreator={}
        % , pdfkeywords={...}
    ]{hyperref}
\else % *** pdflatex output
    \usepackage{times,mathptmx}
    \usepackage[
%       pdftex,
        hyperindex,
        bookmarks,
        bookmarksnumbered=true,
        pdftitle={gnuplot faq},
        pdfauthor={gnuplot},
        pdfsubject={see www.gnuplot.info}
        % , pdfcreator={}
        % , pdfkeywords={...}
    ]{hyperref}
\fi

\usepackage{color}
\definecolor{darkblue}{rgb}{0,0,0.5}
\hypersetup{
  colorlinks   = true, %Colours links instead of ugly boxes
  linkcolor    = darkblue, %Colour of internal links
  urlcolor     = blue %Colour for external hyperlinks
}

% There may be incompatibilities between different versions of
% url.sty, html.sty and hyperref.sty -- it seems there are machines which
% cannot combine them together with simultaneous output to dvi, pdf, html.
% Thus do it this way:
\ifx\html\undefined
    % Modified Dec 2014 (EAM) for use with htlatex or pdflatex
    \def\http#1{{\small\href{http://#1}{\url{http://#1}}}}
    \def\mailto#1{{\small\href{mailto://#1}{\url{mailto://#1}}}}
    \def\news#1{\href{news://#1}{\url{news://#1}}}
    \def\ftp#1#2{\href{ftp://#1#2}{\url{#1} in \url{#2}}}
\else
    % Running this file by latex2html:
    \usepackage{html}
%    \html{
        \newcommand{\news}[1]%
            {\def~{\~{}}\htmladdnormallink{\latex{\url{#1}}\html{\textit{#1}}}%
                {news:#1}%
            }
        \newcommand{\ftp}[2]%
            {\htmladdnormallink{\latex{\url{#1}{} in \url{#2}}%
                    \html{\textit{#1} in \textit{#2}}}%
                {ftp://#1#2}%
            }
        \newcommand{\mailto}[1]%
            {\htmladdnormallink{\latex{\url{<#1>}}\html{\textit{#1}}}%
                {mailto:#1}%
            }
        \newcommand{\http}[1]%
            {\htmladdnormallink{\latex{\url{http://#1}}%
                    \html{\textit{http://#1}}}%
                {http://#1}%
            }
%   }
\fi


% comments and discussions:
% version 1.4 dated 99/10/07
% Revision 1.58 dated 2017/06/04 (last CVS version)
%

%
\newcommand{\gnuplot}{\textbf{gnuplot }}
\newcommand{\Gnuplot}{\textbf{Gnuplot }}


\begin{document}
\title{\gnuplot FAQ}
\author{}
\date{}
\maketitle

\noindent
This material is focused on \gnuplot version 5.
        \\
FAQ version: 15 April 2019


\tableofcontents


\setcounter{section}{-1}
\section{Meta -- Questions}

\subsection{Where do I get this document?}

The newest version of this document is on the web at
\http{www.gnuplot.info/faq/}.

\subsection{Where do I send comments about this document?}

Send comments, suggestions etc to the developer mailing list
\mailto{gnuplot-beta@lists.sourceforge.net}.

\section{General Information}

\subsection{What is \gnuplot?}

\gnuplot is a command-driven plotting program.
It can be used interactively to plot functions and data points in
both two- and three-dimensional plots in many different styles and
many different output formats.  \Gnuplot can also be used as a
scripting language to automate generation of plots.
It is designed primarily for the visual display of scientific data.
\gnuplot is copyrighted, but freely distributable;
you don't have to pay for it. You are welcome to download the source code.


\subsection{How did it come about and why is it called \gnuplot?}

The authors of \gnuplot are:
Thomas Williams, Colin Kelley, Russell Lang, Dave Kotz, John
Campbell, Gershon Elber, Alexander Woo and many others.

The following quote comes from Thomas Williams:
\begin{quote}
     I was taking a differential equation class and Colin was taking
     Electromagnetics, we both thought it'd be helpful to visualize the
     mathematics behind them. We were both working as sys admin for an
     EE VLSI lab, so we had the graphics terminals and the time to do
     some coding. The posting was better received than we expected, and
     prompted us to add some, albeit lame, support for file data.

     Any reference to GNUplot is incorrect. The real name of the program
     is "gnuplot". You see people use "Gnuplot" quite a bit because many
     of us have an aversion to starting a sentence with a lower case
     letter, even in the case of proper nouns and titles. gnuplot is not
     related to the GNU project or the FSF in any but the most
     peripheral sense. Our software was designed completely
     independently and the name "gnuplot" was actually a compromise. I
     wanted to call it "llamaplot" and Colin wanted to call it "nplot."
     We agreed that "newplot" was acceptable but, we then discovered
     that there was an absolutely ghastly pascal program of that name
     that the Computer Science Dept.\ occasionally used. I decided that
     "gnuplot" would make a nice pun and after a fashion Colin agreed.
\end{quote}


\subsection{What does \gnuplot offer?}

\begin{itemize}
\item Two-dimensional functions and data plots combining many different
elements such as points, lines, error bars, filled shapes, labels, arrows, ...
\item Polar axes, log-scaled axes, general nonlinear axis mapping, parametric coordinates
\item Data representations such as heat maps, beeswarm plots, violin plots, histograms, ...
\item Three-dimensional plots of data points, lines, and surfaces in
many different styles (contour plot, mesh)
\item Algebraic computation in integer, float and complex arithmetic
\item Data-driven model fitting using Marquardt-Levenberg minimization
\item Support for a large number of operating systems, graphics
file formats and output devices
\item Extensive on-line help
\item \TeX{}-like text formatting for labels, titles, axes, data points
\item Interactive command line editing and history
\end{itemize}


\subsection{Is \gnuplot suitable for scripting?}

Yes. Gnuplot can read in files containing additional commands during
an interactive session, or it can be run in batch mode by piping a
pre-existing file or a stream of commands to stdin. Gnuplot is used
as a back-end graphics driver by higher-level mathematical packages 
such as Octave and can easily be wrapped in a cgi script for use as a 
web-driven plot generator.  Gnuplot supports context- or data-driven
flow control and iteration using familiar statements
{\em if else continue break while for}.


\subsection{Can I run \gnuplot on my computer?}

\Gnuplot{} is in widespread use on many platforms, including
MS Windows, linux, unix, and OSX.  The current source code retains
supports for older systems as well, including VMS, Ultrix, OS/2, and
MS-DOS. 16-bit platforms are no longer supported.

You should be able to compile the \gnuplot source more or
less out of the box in any reasonably standard (ANSI/ISO C, POSIX)
environment.


\subsection{Legalities}

\Gnuplot{} is authored by a collection of volunteers, who cannot
make any legal statement about the compliance or non-compliance of
\gnuplot or its uses. There is no warranty whatsoever. Use at your own risk.


\subsection{Does \gnuplot have anything to do with the FSF and the GNU
project?}

\Gnuplot{} is neither written nor maintained by the FSF\@. At one time it
was distributed by the FSF but this is no longer true. \Gnuplot{} as a whole
is not covered by the GNU General Public License (GPL).

\Gnuplot{} is freeware in the sense that you don't have to pay for it.
You can use or modify gnuplot as you like however certain restrictions
apply to further distribution of modified versions.
Please read and accept the modification and redistribution terms in
the \textit{Copyright} file.


\subsection{Where do I get further information?}

See the main gnuplot web page \http{www.gnuplot.info}.

Some documentation and tutorials are available in other languages than English.
See \http{gnuplot.sourceforge.net/help.html}, section "Localized learning pages
about gnuplot", for the most up-to-date list.


\section{Setting it up}

\subsection{What is the current version of \gnuplot?}

The current released version of \gnuplot is 5.2, first released in September 2017.
Incremental versions (patchlevel 1, 2, ...) are typically released every six months.
The development version of \gnuplot is currently 5.3.

\subsection{Where can I get \gnuplot?}
\label{where-get-gnuplot}

The best place to start is \http{www.gnuplot.info}. From there
you find various pointers to other sites, including the project
development site on SourceForge \http{sourceforge.net/projects/gnuplot}.

The source distribution ("gnuplot-5.2.6.tar.gz" or a similar name) is
available from the official distribution site \http{sourceforge.net/projects/gnuplot}.

\subsection{Where can I get current development version of \gnuplot?}

The development version of gnuplot is held in a git repository which you
can clone as shown below to inspect or build an executable program 
from the source.

\scriptsize
\begin{verbatim}
  git clone https://git.code.sf.net/p/gnuplot/gnuplot-main gnuplot-gnuplot-main
\end{verbatim}
\normalsize

Important note: questions related to the development version should go to
\mailto{gnuplot-beta@lists.sourceforge.net}.


\subsection{How do I compile \gnuplot on my system?}

Read the release notes and files \textit{README} and \textit{INSTALL}.
You will need C and C++ compilers and installed versions of various
support libraries depending on what configuration options you choose and what
terminal types you want your executable to support.

\begin{itemize}
\item
To build from a release version on linux, use \textit{./configure} (or \textit{./configure {-}{-}prefix=\$HOME/usr}
for an installation for a single user), \textit{make} and finally
\textit{make install}.  Pay close attention to the output from the \textit{configure}
script (yes there is a lot of it).
There you will find hints as to what additional support libraries may be needed and
what additional options may be desirable.  In general you will need to have installed
the "development" package for each of the support libraries.
\item
To build from the development source on linux you must run a separate script \textit{./prepare}
prior to running \textit{./configure} .
\item
On Windows, makefiles can be found in \textit{config/mingw}, \textit{config/msvc},
\textit{config/watcom}, and \textit{config/cygwin}. Update the options in the 
makefile's header and run the appropriate \textit{make} tool in the same directory
as the makefile. Additional instructions can be found in the makefiles.
\item
For other platforms, copy the relevant makefile (e.g. \textit{makefile.os2} for
OS/2) from \textit{config/} to \textit{src/}, optionally update options in the
makefile's header, then change directory to \textit{src} and run \textit{make}.
\end{itemize}


\subsection{What documentation is there, and how do I get it?}

Full documentation is included in the release distribution as a PDF file.
Individual sections can be browsed from inside a gnuplot session
by typing \textit{help {\em keyword}}.
Documentation in other formats can be compiled from source in the
{\em docs} subdirectory.

Online copies in English and Japanese are available at
\http{gnuplot.sourceforge.net/documentation.html}.

\subsection{Worked examples}

There is a directory of worked examples in the the source distribution.
These examples and the resulting plots may also be found online at
\http{gnuplot.sourceforge.net/demo/}.


\subsection{How do I determine which options are compiled into \gnuplot?}

Given that you have a compiled version of \gnuplot, you can use the
\verb+show+ command to display a list of configuration and build options
that were used to build your copy.  The output formats (a.k.a. "terminals")
built into your copy of gnuplot are reported by \verb+set terminal+.

\small
\begin{verbatim}
gnuplot> show version long
gnuplot> set terminal
\end{verbatim}
\normalsize


\section{Working with it.}

\subsection{How do I get help?}

Give the \verb+help+ command at the initial prompt. After that, keep
looking through the keywords. Good starting points are \verb+help plot+
and \verb+help set+.

\begin{itemize}
\item
Read the manual, if you have it.

\item
Run the demos in the {\em demo} subdirectory or look at the 
online copies. They should give you some ideas.

\item
Ask your colleagues, the system administrator or the person who
set up \gnuplot.

\item
There is an old-school usenet group devoted to gnuplot questions
\news{comp.graphics.apps.gnuplot}
inhabited mostly by users who found it last century.

\item
A more browser-oriented help forum may be found on StackOverflow
http{stackoverflow.com/questions/tagged/gnuplot}

\item
If all these fail, post a question to \news{comp.graphics.apps.gnuplot} or send mail
to the gatewayed mailing list \mailto{gnuplot-info@lists.sourceforge.net}.
Please note that due to the overwhelming amount of spam it would otherwise receive,
you must subscribe before you can post to it. Subscription instructions 
may be found at
\http{lists.sourceforge.net/lists/listinfo/gnuplot-info}.

\end{itemize}
When asking a question, always mention the gnuplot version and
operating system you are using.  If you are asking about a plot that 
did not come out the way you expected, please try to show a minimal
set gnuplot commands that produce a plot showing the problem.


\subsection{How do I print out my graphs?}

The kind of output produced by a plot command is determined by a
prior \verb+set terminal+ command.  For non-interactive output
you should pair this with a \verb+set output+ command to provide
a file name.

As an example, the following first plots a graph of sin(x) to the
screen and then redraws that same plot as a PDF output file.
Note: the PDF plot will not look exactly like the plot on the screen.

\small
\begin{verbatim}
gnuplot> plot sin(x)
gnuplot> set terminal pdf
Terminal type is now 'pdfcairo'
Options are ' transparent enhanced fontscale 0.5 size 5.00in, 3.00in '
gnuplot> set output "sin.pdf"
gnuplot> replot
gnuplot> unset output            # close output file (otherwise it stays open)
gnuplot> unset terminal          # return to default interactive terminal
gnuplot>
\end{verbatim}
\normalsize

Using the platform-independent way of restoring terminal by \textit{set term
push/pop} commands, do it by
\small
\begin{verbatim}
gnuplot> set terminal push    # save current terminal type (may not be default)
gnuplot> set terminal pdf
gnuplot> set out 'a.pdf'
gnuplot> replot
gnuplot> unset out
gnuplot> set term pop         # restore saved terminal type
\end{verbatim}
\normalsize

Some interactive terminal types (\textit{win, wxt, qt}) provide a printer icon
on the terminal's toolbar. This tool prints the current plot or saves it to file
using generic system tools rather than by using a different gnuplot terminal type.
That is, the file you get by selecting "save to png" in the print menu may be
different than the file you get from \textit{set term png; replot;}.
In general a plot saved in this way will look more like the screen image than
a plot generated from the command line by changing the terminal type.


\subsection{How do I include my graphs in <word processor>?}

Basically, you save your plot to a file in a format your word
processor can understand (using \verb+set term+ and \verb+set output+,
see above), and then you read in the plot from your word processor. Vector
formats (PostScript, emf, svg, pdf, \TeX{}, \LaTeX{}, etc) should be preferred,
as you can scale your graph later to the right size.

Details depend on which word processor you use; use \verb+set term+ to get a
list of available file formats.

Many word processors can use Encapsulated PostScript (*.eps) for graphs. 
You can generate eps output in \gnuplot using either
\verb+set terminal postscript eps+
or
\verb+set terminal epscairo+
.
\Gnuplot does not embed a bitmap preview image in the output eps file.
To accommodate some word processors you may have to add a preview image yourself.
You can use the GSView viewer for this (available for OS/2, Windows and X11),
or some Unix ps tool.

Some Windows office applications, including OpenOffice.org, can handle
vector images in EMF format. These can be either produced by the emf 
terminal, or by selecting 'Save as EMF...' from the toolbar of the graph
window of the windows terminal.

OpenOffice.org can also read SVG, as well as AutoCAD's dxf format.

There are many ways to use gnuplot to produce graphs for inclusion in a
\TeX\ or \LaTeX\ document.  
Some terminals produce *.tex fragments for direct inclusion; others 
produce *.eps, *.pdf, *.png output to be included using the
\textbackslash{}includegraphics command.
The epslatex and cairolatex terminals produce both a graphics
file (*.eps or *.pdf) and a *.tex document file that refers to it.
The tikz terminal produces full text and graphics to a pdf file
when the output is processed with pdflatex.

Most word processors can import bitmap images (png, pbm, etc).
The disadvantage of this approach is that the resolution of your
plot is limited by the size of the plot at the time it is generated
by gnuplot, which is generally a much lower resolution than the
document will eventually be printed in.


\subsection{How do I edit or post-process a \gnuplot graph?}

This depends on the terminal type you use.

\begin{itemize}

\item \textbf{svg} terminal (scalable vector graphics) output can
be further edited by a svg editor, e.g. 
\textbf{Inkscape} (\http{www.inkscape.org}),
\textbf{Skencil} (\http{www.skencil.org}) or
\textbf{Dia} (\http{projects.gnome.org/dia/}), or loaded
into \textbf{OpenOffice.org} with an on-fly conversion into OO.o Draw
primitives.

\item PostScript or PDF output can be edited directly by tools such
as Adobe Illustrator or Acrobat, or can be converted to a variety
of other editable vector formats by the \textbf{pstoedit} package.
Pstoedit is available at \http{www.pstoedit.net}.

\item The DXF format is the AutoCAD's format, editable by several
other applications.

\item Bitmapped graphics (e.g. png, jpeg, pbm) can be edited using
tools such as ImageMagick or Gimp.  
In general, you should use a vector graphics program to post-process
vector graphic formats, and a pixel-based editing program 
to post-process pixel graphics.

\end{itemize}


\subsection{How do I change symbol size, line thickness and the like?}

Gnuplot offers a variety of commands to set line and point properties,
including color, thickness, point shape, etc.  The command \verb+test+ will
display a test page for the currently selected terminal type showing
the available pre-defined combinations of color, size, shape, etc.
You can use the command \textit{set linetype} to change this or 
define additional combinations.


\subsection{Can I animate my graphs?}

Only one \gnuplot terminal type (gif) directly outputs an animated file:
\begin{verbatim}
set terminal gif animate {delay <time>} {loop <N>} {optimize}
\end{verbatim}

Have a look at
\http{gnuplot.sourceforge.net/demo/animate.html}
in the demo collection.


\subsection{How do I plot implicit defined graphs?}

Implicit graphs or curves cannot be plotted directly in \gnuplot.
However there is a workaround. 
\small
\begin{verbatim}
gnuplot> # An example. Place your definition in the following line:
gnuplot> f(x,y) = y - x**2 / tan(y)
gnuplot> set contour base
gnuplot> set cntrparam levels discrete 0.0
gnuplot> unset surface
gnuplot> set table $TEMP
gnuplot> splot f(x,y)
gnuplot> unset table
gnuplot> plot $TEMP w l
\end{verbatim}
\normalsize
The trick is to draw the single contour line z=0 of the surface
z=f(x,y), and store the resulting contour curve to a temporary file or datablock.
 

\subsection{How to fill an area between two curves}

A plot with filled area between two given curves can be easily obtained using
the pseudo file '+' with \textit{filledcurves closed}. The example below
demonstrates this for two curves f(x) and g(x):
\small
\begin{verbatim}
f(x)=cos(x)
g(x)=sin(x)
xmax=pi/4
set xrange [0:xmax]
plot '+' using 1:(f($1)):(g($1)) with filledcurves
\end{verbatim}
\normalsize

Note that this code fragment fills area between the two curves, regardless of
which one is above the other.  If you want to fill only the area satisfying g(x)<f(x)
or f(x)<g(x)
add an additional keyword \textit{above} or \textit{below} after \textit{filledcurves}.

See the documentation for \textit{help filledcurves}, 
\textit{help special-filenames}, and \textit{help ternary} and see
\textit{fillbetween.dem} in the \textit{demos} directory.


\subsection{Pm3d splot from a datafile does not draw anything}
\label{blank1}

You do \verb+set pm3d; splot 'a.dat'+ and no plot but colorbox appears.
Perhaps there is no blank line in between two subsequent scans (isolines) in
the data file? Add blank lines! If you are curious what this means, then don't
hesitate to look to files like \verb+demo/glass.dat+ or \verb+demo/triangle.dat+
in the gnuplot demo directory.

You can find useful the following awk script (call it e.g. \verb+addblanks.awk+)
which adds blank lines to a data file whenever number in the first column
changes:
\small
\begin{verbatim}
/^[[:blank:]]*#/ {next} # ignore comments (lines starting with #)
NF < 3 {next} # ignore lines which don't have at least 3 columns
$1 != prev {printf "\n"; prev=$1} # print blank line
{print} # print the line
\end{verbatim}
\normalsize
Then, either preprocess your data file by command
\verb+awk -f addblanks.awk <a.dat+ or plot the datafile under a unixish platform
by \verb+gnuplot> splot "<awk -f addblanks.awk a.dat"+.


\subsection{Drawing 2D projection of 3D data}

Use \textit{set view map}
There are also plotting styles \verb+with image+ and \verb+with rgbimage+
for plotting 2D color images.

\subsection{How to overlay dots/points scatter plot onto a pm3d map/surface}

Use the explicit (see also implicit) switch of the pm3d style:
\small
\begin{verbatim}
gnuplot> set pm3d explicit
gnuplot> splot x with pm3d, x*y with points
\end{verbatim}
\normalsize


\subsection{How to produce labeled contours}

Labeling individual contours in a contour plot used to require special
tricks and extra processing steps in \gnuplot version 4.
See
\http{gnuplot.sourceforge.net/scripts/index.html\#tricks-here}.

In version 5 the procedure is much simpler.  Plot the contours twice,
once "with lines" and once "with labels".  To make the labels stand out
it may help to use 
\begin{verbatim}
set style textbox opaque noborder
set contours
splot 'DATA' with line, 'DATA' with labels boxed
\end{verbatim}

\subsection{Color facets with pm3d}

It is possible to draw colors facets of a 3D objects, organized in such a file:
\small
\begin{verbatim}
# triangle 1
x0 y0 z0 <c0>
x1 y1 z1 <c1>

x2 y2 z2 <c2>
x2 y2 z2 <c2>


# triangle 2
x y z
...
\end{verbatim}
\normalsize

Notice the positioning single and double blank line. \textit{<c>} is an optional
color.

Then plot it by (either of splot's):
\small
\begin{verbatim}
set pm3d
set style data pm3d
splot 'facets.dat'
splot 'facets_with_color.dat' using 1:2:3:4
\end{verbatim}
\normalsize

Note that you avoid surface lines by \textit{set style data pm3d} or 
\textit{splot ... with pm3d}.

In the above example, pm3d displays triangles as independent surfaces.
They are plotted one surface after another, as found in the data file.
Parts overlapping in 2D projection are overdrawn.

Gnuplot is not 3D modeling program. Its hidden routines apply for points and
lines, but not for faces. 
Without handling the data as a collection of faces, there would be no surface
anything could be hidden behind.  The 'hidden3d' algorithm works by using the
input data in two ways: first, to set up a collection of triangles (made from a
mesh of quadrangles) that form the surface, second as a collection of edges.  It
then goes through all those edges, checking what parts of them are not hidden
behind any faces, and draws those.

Consequently, gnuplot won't draw your surface or 3D object as a virtual reality.
It works OK for \textit{set pm3d map} but for true 3D you would be probably more
happy writing a converter of your facets into a VRML file.


\subsection{Palette for printing my color map on color as well as black\&white
printer?}

Try \textit{set palette cubehelix}.


\section{Wanted features}

\subsection{What's new in \gnuplot 5.0, 5.2, 5.2.5 etc?}

Too many things to list here.
Please refer to the {\em Release Notes} and the \textit{NEWS} file in the
source distribution, or the "New features" section in the gnuplot documentation.


\subsection{Does \gnuplot support a driver for <graphics format>?}

To see a list of the available graphic drivers for your installation of
\gnuplot, type \verb+set term+.

Some graphics drivers are included in the normal distribution
but are not built by default. If you want to use them, you'll
have to recompile from source.


\subsection{Does \gnuplot have hidden line removal?}

Yes.

\subsection{Does \gnuplot support bar-charts/histograms/boxes?}

\Gnuplot{} supports various clustered and stacked histogram styles to display
pre-tabulated data.  It also offers a few options for accumulating raw data
into bins, which can in turn be displayed as a bar chart.  See the documentation
for \textit{bins} and for \textit{smooth frequency}.

\subsection{Does \gnuplot support pie charts? quarterly time charts?}

Pie charts are sort of difficult in \gnuplot, but see
\http{gnuplot.sourceforge.net/demo/circles.html}, 
or have a look at
\http{gnuplot-tricks.blogspot.com/2009/08/pie-charts-entirely-in-gnuplot.html}.

The demo collection contains an example of a simple Gantt chart.

\subsection{Can I arrange multiple plots next to each other on a single page?}

Yes. \verb+set multiplot+.

\subsection{Does \gnuplot support multiple y-axes on a single plot?}

Yes. 2D plots can have separate x axes at the bottom (x1) and top (x2),
and separate y axes at the left (y1) and right (y2).  Version 5 offers a
plot mode \textit{with parallelaxes} that allows additional y axes to be
defined.


\subsection{Can I put both commands and data into a single file?}

Version 5 supports named blocks of data in "here document" format:
\small
\begin{verbatim}
gnuplot> $DATABLOCK << EOD
cats 4 2
dogs 1 4
EOD
gnuplot> plot $DATABLOCK using 2:3:1 with labels
\end{verbatim}
\normalsize
Once the named block has been defined, it can be used as many times
as you like.

Data can also be provided in-line as part of a plot command using the
pseudo-file \verb+plot "-"+.  In this case the data can only be used
once.

\small
\begin{verbatim}
gnuplot> plot "-"
1 1
2 4
3 9
e
\end{verbatim}
\normalsize


\subsection{Can I put superscripts/subscripts in my labels?}

Most terminal types (gnuplot's name for the choice of output device) support
an "enhanced text" mode.  This lets you use sub- and superscripts, italic and
boldface, and different font faces or font sizes.

\subsection{How to use Greek letters or other special symbols?}

The old-style way is to use enhanced text mode to switch to a specialize font,
e.g. the Adobe "Symbol" font, that maps the characters you want onto ascii
letters 'a', 'b', etc.  This may still be necessary for PostScript output.
However a much simpler way is to select UTF-8 encoding and enter the
special characters just as you would any other text.
This obviates the need to change fonts and gives you access to all unicode
code points including CJK character sets.  To actually print or view the
files produced by \gnuplot you still need appropriate fonts installed on
your computer or output device.  \Gnuplot itself does not provide fonts.

The various \LaTeX{} terminal types (\textit{latex, epslatex, tikz, context, cairolatex})
hand off text generation to \LaTeX{}. In this case you can use normal \LaTeX{}
markup like \verb+"\\alpha_{3}"+ or \verb+'\alpha_{3}'+ .

\subsection{How do I include accented characters}
% \subsection{Can I type labels in Czech, French, Hungarian, Russian...}

To obtain accented characters like \"u or \^n in your labels you should use
8bit character codes together with the appropriate encoding option.
See the following example:

\small
\begin{verbatim}
gnuplot> set encoding iso_8859_1
gnuplot> set title "M\374nchner Bierverbrauch \374ber die Jahre"
gnuplot> plot "bier.dat" u 1:2
\end{verbatim}
\normalsize

Consequently, you can type labels in Czech, French, Hungarian, Russian... by
means of an appropriate \textit{set encoding}. However, you cannot mix two
encodings in one file (e.g. accents for west and east latin encodings).

A more general solution is to use UTF-8 encoding as described above, giving
access to any accented characters with a unicode code point.

\subsection{Can I do 1:1 scaling of axes?}

Try \verb+set size square+
or \verb+set view equal xy+.


\subsection{Can I put different text sizes into my plots?}

Most terminal types allow you to specify a starting font face and size.
The "enhanced text" mode allows you to change fonts, text sizes, bold and
italic styles within a plot.

\subsection{How do I skip data points?}

By specifying \textit{?} as a data value, as in
\small
\begin{verbatim}
        1 2
        2 3
        3 ?
        4 5
\end{verbatim}
\normalsize

See also \textit{set missing}.
See also \textit{set datafile commentschars} for specifying comment characters in
data files.


\subsection{How do I plot every nth point?}

This can be specified with various options for the command \verb+plot+,
for example \verb+plot 'a.dat' every 2+.  If you want to draw a line 
through every point but only draw a point symbol at every nth point,
then try \verb+plot 'a.dat' with linespoints pointinterval n+.


\subsection{How do I plot a vertical line?}

Depending on context, the main methods are:
\begin{itemize}
\item \verb+set arrow .... .... nohead+ where you have to compute
explicitly the start and the end of the arrow.
\item generate (inlined) datapoints and plot them
\end{itemize}


\subsection{How do I plot data files}

Easily: by a command \textit{plot 'a.dat'}. In 3D, use \textit{splot 'a.dat'} --
but don't forget to put a blank line in between two subsequent scans (isolines),
otherwise you will get an error that the data is not gridded; see also question 
\ref{blank1}. If your data are not gridded, then use \textit{set dgrid3d \{many
options\}}.


\subsection{How do I replot multiplot drawing}

You cannot directly: gnuplot supports \verb+replot+ command, not
\verb+remultiplot+. You have to write the complete sequence of commands since
\verb+set multiplot+ till \verb+unset multiplot+ into a script file. Then
you can \verb+load+ the script into gnuplot as many times as you need for
replotting the drawing to different terminals or output files.


\section{Miscellaneous}

\subsection{I've found a bug, what do I do?}

First, try to see whether it actually is a bug, or whether it
is a feature which may be turned off by some obscure set--command.

Next, see whether you have an old version of \gnuplot; if you do,
chances are the bug has been fixed in a newer release.

The development version may already contain fixes for bugs reported
since the release of the current version.
Before submitting a bug report, please check whether the bug in question
has already been fixed.

If, after checking these things, you still are convinced that there is a
bug please report it using the bug-tracker at
\http{sourceforge.net/p/gnuplot/bugs}.
Be sure to include the version of \gnuplot (including patchlevel) and 
the operating system you are running it on.
It helps a lot if you can provide a simple script that reproduces the error.

The tracker on sourceforge is for reporting bugs and collecting bug fixes
that will appear in a subsequent release.
The various online forums are a better place to ask about
work arounds or actually solving \gnuplot related problems.


\subsection{Can I use \gnuplot routines for my own programs?}

On systems supporting pipes, you can pipe commands to \gnuplot from other
programs. Many applications with gnuplot as the graphics engine, like Octave
(\http{www.octave.org}), uses this method. This also works from a cgi script to
drive \gnuplot from a forms-based web page.

\subsection{What extensions have people made to \gnuplot? Where can I get
them?}

Extensions or patches are available on the "Patches" page of the
gnuplot development site
\http{sourceforge.net/p/gnuplot/patches/}.
The current development version will generally include features that are
not yet part of the most recent official release of gnuplot.

\subsection{I need an integration, fft, iir-filter,...!}

\Gnuplot{} has been and is a plotting program, not a data
processing or mathematical program suite. Therefore \gnuplot
can't do that. Look into the demo file "bivariat.dem" for a basic
implementation of an integration.  However \gnuplot version 5 does
support calling functions from a dynamically loaded external shared
object, i.e. a plugin.  So if you want to code up some complicated
function in C or another language with compatible calling conventions,
you can compile it into a plugin for \gnuplot to import.

\subsection{Can I do heavy-duty data processing with \gnuplot? or
What is beyond \gnuplot?}

\gnuplot by itself is not suited very well for heavy numerical computation. 
On the other hand it can handle very large data sets.

One thing you might try is \textit{fudgit}, an interactive multi-purpose
fitting program written by Martin-D. Lacasse.
It can use \gnuplot as its graphics back end.

You might also want to look at the applications developed by
the Software Tools Group (STG) at the National Center for
Supercomputing Applications \http{ncsa.uiuc.edu}.

You can also try pgperl, an integration of the PGPLOT plotting
package with Perl 5. Information can be found at
\http{www.ast.cam.ac.uk/AAO/local/www/kgb/pgperl}, the source is
available from \ftp{ftp.ast.cam.ac.uk}{/pub/kgb/pgperl/} or
\ftp{linux.nrao.edu}{/pub/packages/pgperl/}.

Another possibility is \textbf{Octave}. To quote from its README: Octave is a
high-level language, primarily intended for numerical computations. Octave is
licensed under GPL, and in principle, it is a free Matlab clone. It provides a
convenient command line interface for solving linear and nonlinear problems
numerically. The latest released version of Octave is always available from
\http{www.octave.org}. By the way, octave uses \gnuplot as its plotting
engine, so you get a data-processing program on top of \gnuplot.

Finally there is \textit{scilab} (\http{www.scilab.org}), an open source
alternative to \textit{matlab}.


\subsection{How to use hotkeys in my interactive terminals}

Most of the interactive terminals support both pre-defined and user-defined
hotkeys to replot, toggle plot elements, change axis scaling, and so on.
Hit \textit{h} in an active gnuplot plot window to get list of hotkeys. 
Read \textit{help mouse} and \textit{help bind} for more information.


\subsection{I have ported \gnuplot to another system, or patched it. What
do I do?}

The preferred way of submitting, commenting and upgrading patches is
via 'Patches' section on 
\http{sourceforge.net/p/gnuplot/patches/}.
You may want to send a note to \mailto{gnuplot-beta@lists.sourceforge.net} for
more lively discussion.


\subsection{I want to help in developing the next version of \gnuplot.
What can I do?}

Join the \gnuplot beta test mailing list by sending a mail
containing the line
\verb+subscribe gnuplot-beta+
in the body (not the subject) of the mail to
\mailto{Majordomo@lists.sourceforge.net}.


\subsection{Open questions for inclusion into the FAQ?}


% \mailto{gnuplot-beta@lists.sourceforge.net}.

Please submit your questions (along with the answer) to 
\mailto{gnuplot-beta@lists.sourceforge.net}.


\section{Making life easier}

\subsection{How do I plot two functions in non-overlapping regions?}

This used to be complicated.  In version 5 it is easy - place the
desired range immediately before each function being plotted.
For example, to plot experimental data and two different functional
models f1 and f2 covering two different portions of the domain:
\small
\begin{verbatim}
gnuplot> set autoscale x  # get x range from the data
gnuplot> plot "data", [-100:0] f1(x), [0:100] f2(x)
\end{verbatim}
\normalsize


\subsection{How do I run my data through a filter before plotting?}

If your system supports the popen() function, as Unix does, you
should be able to run the output through another process, for
example a short awk program, such as

\small
\begin{verbatim}
gnuplot> plot "< awk ' { print $1, $3/$2 } ' file.in"
\end{verbatim}
\normalsize

The plot command is very powerful and is able to do some
arithmetic on datafiles. See \verb+help plot+.

The above filtering works seamlessly under Unixes and OS/2. On Windows, this
is only supported by default in \textit{gnuplot} version 5, but required a 
separate program \textit{wgnuplot\_pipes} in some earlier versions.

\subsection{How do I save and restore my current settings?}

Use the \verb+save+ and \verb+load+ commands for this; see \verb+help save+
and \verb+help load+ for details.

You can save the current terminal and restore it later without touching the
filesystem by \textit{set term push} and \textit{set term pop}, respectively.


\subsection{How do I plot lines (not grids) using splot?}

If the data input to splot is arranged such that each line contains
the same number of data points (using blank lines as delimiters, as usual),
splot will by default treat the data as describing a surface.
If you want to draw individual lines instead, try some combination of
\textit{unset surface}, \textit{set surface explicit}, \textit{plot ... nosurface}.


\subsection{How do I plot a function f(x,y) that is bounded by other
          functions in the x-y plane?}

Here is one way:
\small
\begin{verbatim}
gnuplot> f(x,y) = x**2 + y **2
gnuplot> x(u) = 3*u
gnuplot> yu(x) = x**2
gnuplot> yl(x) = -x**2
gnuplot> set parametric
gnuplot> set cont
gnuplot> splot [0:1] [0:1] u,yl(x(u))+(yu(x(u)) - yl(x(u)))*v,\
> f(x(u), (yu(x(u)) - yl(x(u)))*v)
\end{verbatim}
\normalsize


\subsection{How do I call \gnuplot from my own programs?}

On unix-like systems, commands to gnuplot can be piped via stdin.
Output from \gnuplot's \verb+print+ command can be read via a named pipe.
On Windows, due to the lacking standard input (stdin) in GUI programs,
you either need to use the console version \textit{gnuplot} (recommended),
or use \textit{wgnuplot\_pipes}, which has an additional console window
attached. The old helper program \textit{pgnuplot} is still included
in the distribution package.


\subsection{What if I need h-bar (Planck's constant)?}

The most straightforward way is to use a UTF-8 font, and type in the
$\hbar$ character (Unicode code point \#x210F) directly. 
 
This does not work in PostScript, however, so you must use approximations
like
\verb+ @{/=56 -} {/=24 h}+ or
\verb+ {/=8 @{/Symbol=24 -} _{/=14 h}}+
In the latter, the "-" (a long one in /Symbol) is non-spacing and 24-pt.
The 14-pt "h" is offset by an 8-pt space (which is the space preceding
the "\_") but smaller, since it's written as a subscript.
But these don't look too much like the hbar we're used to, since the bar
is horizontal instead of sloped.  I don't see a way to get that.  I
tried using an accent (character 264 in iso-latin-1 encoding), but I haven't found a
way to scale and position the pieces correctly. 
One more possibility would be \verb+{/=14 @^{/Symbol=10 -}{/=14 h}}+.

The reduced Planck's constant can be set very easily by using the
AMS-LaTeX PostScript fonts which are available from
\http{www.ams.org/tex/amsfonts.html} (also included in many LaTeX
distributions). \Gnuplot{} (see \verb+help fontpath+) and the
PostScript interpreter (usually Ghostscript) have to know where the
file \verb+ msbm10.pfb+ (or \verb+ msbm10.pfa+) resides. Use
\verb+ {/MSBM10 \175}+ to produce \verb+ \hslash+ which is a "h"
superimposed by a sloped bar. The standard \verb+ \hbar+ (horizontal
bar) has the octal code 176. Please note that h-bar exists only as an
italic type.

\subsection{What if I need the Solar mass symbol?}

As with Planck's constant, the most straightforward way is to use a 
UTF-8 font, and type in the $\odot$ character (Unicode code point \#x2299 ; "circled dot operator") directly. 
The very similar glyph at code point \#x2609 ; "sun" may be even better, but not many fonts have it.
 

\subsection{How do I give exact positions for the graph borders on the page?}

Specify the position of the top, bottom, left, and right borders in
terms of their fractional position within the page:

\small
\begin{verbatim}
set lmargin at screen 0.05
set bmargin at screen 0.05
set rmargin at screen 0.95
set tmargin at screen 0.95
\end{verbatim}
\normalsize


\section{Common problems}


\subsection{Help! None of my fonts work.}

Gnuplot does not do font handling by itself; it must necessarily leave
that to the individual device support libraries. Unfortunately, this
means that different terminal types need different help in finding
fonts. Here are some quick hints. For more detailed information please
see the gnuplot documentation for the specific terminal type you are
having problems with.

\begin{description}
\item [{png/jpeg/gif}] These terminal types use the libgd support library, which
searches for fonts in the directories given in the environmental variable
GDFONTPATH. Once you get libgd fontpaths sorted out, you will probably
want to set a default font for gnuplot.
For example: \verb+setenv GNUPLOT_DEFAULT_GDFONT verdana+
\item [{post}] PostScript font names are not resolved until the document
is printed. Gnuplot does not know what fonts are available to your
printer, so it will accept any font name you give it. However, it
is possible to bundle a font with the gnuplot output; please see the
instructions given by gnuplot's internal command {}``help set term
post fontfile''.
\item [{svg}] Font handling is viewer-dependent. 
\item [{x11}] The x11 terminal uses the normal x11 font server mechanism.
The only tricky bit is that in order to use multi-byte fonts you must
explicitly say so:
\small\begin{verbatim}
set term x11 font "mbfont:sazanami mincho,vera,20"
\end{verbatim}\normalsize
\item [{win}] Select "Choose font..." from the "Options" pull-down menu 
in the toolbar.
\item [{wxt, qt}] On linux systems these terminals rely on fonts provided
by the system's \textit{fontconfig} utility.
\end{description}

\subsection{\Gnuplot{} does not open a plot window on VMS. Why?}
On VMS, you need to make several symbols:

\small
\begin{verbatim}
        $ gnuplot_x11 :== $disk:[directory]gnuplot_x11
        $ gnuplot :== $disk:[directory]gnuplot.exe
        $ def/job GNUPLOT$HELP disk:[directory]gnuplot.hlb
\end{verbatim}
\normalsize

Then run \gnuplot from your command line, and use
\verb+set term x11+.

On Unix systems the x11 and qt terminals require installation of
separate helper programs \textit{gnuplot\_x11} and \textit{gnuplot\_qt}.
These are usually installed in a directory \textit{/usr/libexec/gnuplot/5.0/}
and \gnuplot knows to look for them there.  If they are installed somewhere
else or gnuplot is having trouble finding them, try setting the environmental
variable \verb+GNUPLOT_DRIVER_DIR+.


\subsection{Why does \gnuplot ignore my very small numbers?}


For some purposes \Gnuplot{} treats numbers less than 1e-08 as being zero.
Thus, if you are trying to plot a collection of very small
numbers, they may be plotted as zero. Worse, if you're plotting
on a log scale, they will be off scale. Or, if the whole set of
numbers is "zero", your range may be considered empty:

\small
\begin{verbatim}
gnuplot> plot 'test1'
Warning: empty y range [4.047e-19:3e-11], adjusting to [-1:1]
gnuplot> set yrange [4e-19:3e-11]
gnuplot> plot 'test1'
              ^
y range is less than `zero`
\end{verbatim}
\normalsize

The solution is to change \gnuplot's idea of "zero":
\small
\begin{verbatim}
gnuplot> set zero 1e-20
\end{verbatim}
\normalsize

For more information, type \verb+help set zero+.


\subsection{When I run \gnuplot from the shell or a script, the resulting plot flashes by on the screen and then disappears}

\begin{enumerate}
\item Put a \verb+pause -1+ after the plot command in the file, or at the file end.

\item Use command \verb+gnuplot filename.gp -+ (yes, dash is the last
parameter) to stay in the interactive regime when the script completes.

\item Run \gnuplot as \textit{gnuplot -persist}

\item On Windows you can also use either \verb+-persist+ or \verb+/noend+.

\item Give the \textit{persist} option as part of the \textit{set terminal} command.
\end{enumerate}



\subsection{My formulas (like 1/3) are giving me nonsense results! What's going on?}

\Gnuplot{} does integer, and not floating point, arithmetic on
integer expressions. For example, the expression 1/3 evaluates
to zero. If you want floating point expressions, supply
trailing dots for your floating point numbers. Example:


\small
\begin{verbatim}
gnuplot> print 1/3
                0
gnuplot> print 1./3.
                0.333333
\end{verbatim}
\normalsize

This way of evaluating integer expressions is shared by both C and Fortran.


\subsection{My output files are incomplete!}

You may need to flush the output with a closing \verb+set output+.
Some output formats (postscript, pdf, latex, svg, ...) can include several
pages of plots in a single output file.  For these output modes, gnuplot
leaves the file open after each plot so that you can add additional plots
to it.  The file is not completed and made available to external applications
until you explicitly close it (\verb+set output+ or \verb+unset output+),
or select a different terminal type (\verb+set term+) or exit gnuplot.
Output formats that contain only a single 'page' (png, emf, ...)
should not suffer from this problem.


\subsection{Calling \gnuplot in a pipe or with a \gnuplot-script
doesn't produce a plot!}

You can call \gnuplot by using a short Perl-script like the
following:
\small
\begin{verbatim}
#!/usr/local/bin/perl -w
open (GP, "|/usr/local/bin/gnuplot -persist") or die "no gnuplot";
# force buffer to flush after each write
use FileHandle;
GP->autoflush(1);
print GP,"set term x11;plot '/tmp/data.dat' with lines\n";
close GP
\end{verbatim}
\normalsize

\Gnuplot{} closes its plot window on exit. The \verb+close GP+
command is executed, and the plot window is closed even before you have
a chance to look at it.

There are three solutions to this: first, use the \verb+pause -1+
command in \gnuplot before closing the pipe. Second, close the pipe
only if you are sure that you don't need \gnuplot and its plot window
anymore. Last, you can use the command line option \verb+-persist+: this
option leaves the X-Window System plot window open.


\section{Credits}

\Gnuplot{} 3.7's main contributors are (in alphabetical order)
Hans-Bernhard Broeker, John Campbell, Robert Cunningham, David Denholm,
Gershon Elber, Roger Fearick, Carsten Grammes, Lucas Hart, Lars Hecking,
Thomas Koenig, David Kotz, Ed Kubaitis, Russell Lang, Alexander Lehmann,
Alexander Mai, Carsten Steger, Tom Tkacik, Jos Van der Woude, James R.
Van Zandt, and Alex Woo.  Additional substantial contributors to version 4.0
include Ethan Merritt, Petr Mikul\'{\i}k and Johannes Zellner.
Version 4.2, 4.4, 4.6 and 5.0 releases were coordinated by Ethan Merritt.

This list was initially compiled by John Fletcher with contributions
from Russell Lang, John Campbell, David Kotz, Rob Cunningham, Daniel
Lewart and Alex Woo. Reworked by Thomas Koenig from a draft
by Alex Woo, with corrections and additions from Alex Woo, John
Campbell, Russell Lang, David Kotz and many corrections from Daniel Lewart.
Again reworked for \gnuplot 3.7 by Alexander Mai and Juergen v.Hagen
with corrections by Lars Hecking, Hans-Bernhard Broecker and others.
Revised for \gnuplot version 4 by Petr Mikul\'{\i}k and Ethan Merritt.
Revised for \gnuplot version 5 by Ethan Merritt.


\end{document}
%%
